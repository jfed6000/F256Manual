\chapter{System Control Registers}
\label{sec:sysctrl}

\section*{The Buzzer and Status LEDs}

The \jr\ has several software-controllable LEDs. There are the SD card access LED and the power LED, but there are also two status LEDs on the board which may be controlled either manually or set to flash automatically. All the LEDs under ``manual'' control can be controlled by setting or clearing their relevant flags in the SYS0 register (0xD6A0) (see table:~\ref{tab:sys_ctrl_reg}). The power LED is controlled by PWR\_LED. The SD card LED is controlled by SD\_LED.

\begin{table}[ht]
    \begin{center}
        \begin{tabular}{|c|c|c|c|c|c|c|c|c|c|c|} \hline
            Address & R/W & Name & 7 & 6 & 5 & 4 & 3 & 2 & 1 & 0 \\\hline\hline

            \verb+0xD6A0+ & R/W & SYS0 & RESET & SD\_WP & SD\_CD & BUZZ & L1 & L0 & SD\_LED & PWR\_LED \\ \hline
            \verb+0xD6A1+ & R/W & SYS1 & \multicolumn{2}{|c|}{L1\_RATE}
                                       & \multicolumn{2}{|c|}{L0\_RATE}
                                       & \multicolumn{2}{|c|}{---}
                                       & L1\_MAN & L0\_MAN \\ \hline
        \end{tabular}
    \end{center}
    \caption{System Control Registers}
    \label{tab:sys_ctrl_reg}
\end{table}

The two status LEDs on the board are a little more complex. They may be in manual or automatic mode. The two flags L0\_MAN and L1\_MAN in SYS1 control which mode they are in. If an LED's flag is clear (0), then the LED is under manual control and its equivalent flag in SYS0 controls whether the LED is on or off. If the flag is set, then the LED is set to flash automatically, and the LED's flashing rate will be set by pair of bits L0\_RATE or L1\_RATE according to table~\ref{tab:led_rates}.

For the PC speaker, there is the BUZZ flag. By toggling BUZZ, a program can tweak the speaker and make a noise.

\begin{table}[ht]
    \begin{center}
        \begin{tabular}{|c|c|c|} \hline
            RATE1 & RATE0 & Rate \\\hline\hline
            0 & 0 & 1s \\ \hline
            0 & 1 & 0.5s \\ \hline
            1 & 0 & 0.4s \\ \hline
            1 & 1 & 0.2s \\ \hline
        \end{tabular}
    \end{center}
    \caption{LED Flash Rates}
    \label{tab:led_rates}
\end{table}

\section*{Software Reset}

A program can trigger a system reset. This can be done by writing the value \verb+0xDE+ to \verb+0xD6A2+ and the value \verb+AD+ to \verb+0xD6A3+ to validate that a reset is really intended (see table:~\ref{tab:sys_reset}), and then setting the most significant bit (RESET) of \verb+0xD6A0+ to actually trigger the reset.

\begin{table}[ht]
    \begin{center}
        \begin{tabular}{|c|c|c|c|c|c|c|c|c|c|c|} \hline
            Address & R/W & Name & 7 & 6 & 5 & 4 & 3 & 2 & 1 & 0 \\\hline\hline
            \verb+0xD6A2+ & R/W & RST0 & \multicolumn{8}{|c|}{Set to 0xDE to enable software reset} \\ \hline
            \verb+0xD6A3+ & R/W & RST1 & \multicolumn{8}{|c|}{Set to 0xAD to enable software reset} \\ \hline
        \end{tabular}
    \end{center}
    \caption{System Reset}
    \label{tab:sys_reset}
\end{table}

\section*{Random Numbers}

The \jr\ has a built-in pseudo-random number generator that produces 16-bit random numbers (see table:~\ref{tab:rng_reg}). To use the random number generator, a program just sets the enable flag and then reads the random numbers from RNDL and RNDH (0xD6A4 and 0xD6A5). The program can set the seed value to better randomize the numbers by storing a seed value in those same locations and then toggling SEED\_LD (set to load the seed value then reclear).

\begin{table}[ht]
    \begin{center}
        \begin{tabular}{|c|c|c|c|c|c|c|c|c|c|c|} \hline
            Address & R/W & Name & 7 & 6 & 5 & 4 & 3 & 2 & 1 & 0 \\\hline\hline
            \verb+0xD6A4+ & W & SEEDL & \multicolumn{8}{|c|}{SEED[7\ldots 0]} \\ \hline
            \verb+0xD6A4+ & R & RNDL & \multicolumn{8}{|c|}{RND[7\ldots 0]} \\ \hline
            \verb+0xD6A5+ & W & SEEDH & \multicolumn{8}{|c|}{SEED[15\ldots 0]} \\ \hline
            \verb+0xD6A5+ & R & RNDH &  \multicolumn{8}{|c|}{RND[15\ldots 0]} \\ \hline

            \verb+0xD6A6+ & W & RND\_CTRL & \multicolumn{6}{|c|}{---} & SEED\_LD & ENABLE \\ \hline
            \verb+0xD6A6+ & R & RND\_STAT & DONE & \multicolumn{7}{|c|}{---} \\ \hline

        \end{tabular}
    \end{center}
    \caption{Random Number Generator}
    \label{tab:rng_reg}
\end{table}

\begin{description}
    \item[ENABLE] set to turn on the random number generator

    \item[SEED\_LD] set to load a value stored in SEEDL and SEEDH as the seed value for the random number generator

    \item[RNDL and RNDH] read 16-bit random numbers from these registers when the random number generator is enabled
\end{description}

\section*{Machine ID and Version Information}

Nine registers are set aside to identify the machine, the version of the printed circuit board, and the version of the FPGA. See table~\ref{tab:machine_id_ver} for the various registers. All of the registers are read-only, and only the chip information will change over the course of the machine's life span. The machine ID contains a four-bit code that is common between all the Foenix machines (see table~\ref{tab:machine_ids}).

For the \jr, the machine ID will be 2.

\begin{leftbar}
	NOTE: The \jr\ RevA board does not support the PCB major and minor revision number registers or the PCB date code registers.
\end{leftbar}

\begin{table}[ht]
    \begin{center}
        \begin{tabular}{|c|c|c|c|c|c|c|c|c|c|c|} \hline
            Address & R/W & Name & 7 & 6 & 5 & 4 & 3 & 2 & 1 & 0 \\\hline\hline
            \verb+0xD6A7+ & R & MID & \multicolumn{3}{|c|}{---} & \multicolumn{5}{|c|}{ID} \\ \hline\hline

            \verb+0xD6A8+ & R & PCBID0 & \multicolumn{8}{|c|}{ASCII character 0: ``A''} \\ \hline
            \verb+0xD6A9+ & R & PCBID1 & \multicolumn{8}{|c|}{ASCII character 1: ``0''} \\ \hline
            \verb+0xD6EB+ & R & PCBMA & \multicolumn{8}{|c|}{PCB Major Rev (ASCII)} \\ \hline
            \verb+0xD6EB+ & R & PCBMI & \multicolumn{8}{|c|}{PCB Minor Rev (ASCII)} \\ \hline
            \verb+0xD6EB+ & R & PCBD & \multicolumn{8}{|c|}{PCB Day (BCD)} \\ \hline
            \verb+0xD6EB+ & R & PCBM & \multicolumn{8}{|c|}{PCB Month (BCD)} \\ \hline
            \verb+0xD6EB+ & R & PCBY & \multicolumn{8}{|c|}{PCB Year (BCD)} \\ \hline\hline

            \verb+0xD6AA+ & R & CHSV0 & \multicolumn{8}{|c|}{Chip subversion in BCD (low)} \\ \hline
            \verb+0xD6AB+ & R & CHSV1 & \multicolumn{8}{|c|}{Chip subversion in BCD (high)} \\ \hline
            \verb+0xD6AC+ & R & CHV0 & \multicolumn{8}{|c|}{Chip version in BCD (low)} \\ \hline
            \verb+0xD6AD+ & R & CHV1 & \multicolumn{8}{|c|}{Chip version in BCD (high)} \\ \hline
            \verb+0xD6AE+ & R & CHN0 & \multicolumn{8}{|c|}{Chip number in BCD (low)} \\ \hline
            \verb+0xD6AF+ & R & CHN1 & \multicolumn{8}{|c|}{Chip number in BCD (high)} \\ \hline
        \end{tabular}
    \end{center}
    \caption{Machine ID and Versions}
    \label{tab:machine_id_ver}
\end{table}

\begin{table}[ht]
    \begin{center}
        \begin{tabular}{|c|c|c|c|c|c|} \hline
            MID4 & MID3 & MID2 & MID1 & MID0 & Machine \\\hline\hline
            0 & 0 & 0 & 0 & 0 & C256 FMX \\ \hline
            0 & 0 & 0 & 0 & 1 & C256 U \\ \hline
            0 & 0 & 0 & 1 & 0 & F256jr \\ \hline
            1 & 0 & 0 & 1 & 0 & F256K \\ \hline
            0 & 0 & 0 & 1 & 1 & A2560 Dev \\ \hline
            0 & 0 & 1 & 0 & 0 & Gen X \\ \hline
            0 & 0 & 1 & 0 & 1 & C256 U+ \\ \hline
            0 & 0 & 1 & 1 & 0 & Reserved \\ \hline
            0 & 0 & 1 & 1 & 1 & Reserved \\ \hline
            0 & 1 & 0 & 0 & 0 & A2560 X \\ \hline
            0 & 1 & 0 & 0 & 1 & A2560 U \\ \hline
            0 & 1 & 0 & 1 & 0 & A2560 M \\ \hline
            0 & 1 & 0 & 1 & 1 & A2560 K \\ \hline
        \end{tabular}
    \end{center}
    \caption{Machine IDs}
    \label{tab:machine_ids}
\end{table}
