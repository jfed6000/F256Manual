\chapter{The Text Screen}

The display on the \jr\ is managed by TinyVicky, which is the smaller member of the Vicky family of display controllers in the other Foenix machines. TinyVicky provides several display engines to let your programs control the screen:

\begin{itemize}
    \item Text: an old school style text screen where the characters to display are stored in a text matrix, and the shape of those characters comes from font memory. Text mode characters are 8 pixels wide by 8 pixels high.
    \item Bitmap: a simple pixel graphics mode that can be either 300x240 or 300x200.
    \item Sprite: an engine to display small, movable sprites on the screen.
    \item Tile: an engine to display images on the screen made up of tiles from a tile set.
\end{itemize}

The bitmap, sprite, and tile engines are considered graphics modes. TinyVicky will let you display either text by itself, a mix of the graphics modes by themselves, or text overlayed on top of the graphics modes.

\section{Text Matrix}

The memory for the characters to display on the screen is the text matrix, which is stored in I/O page 2. When this I/O page is swapped into the CPU address space, it appears at 0xC000. Each byte of memory corresponds to a single character on the screen in left to right, top to bottom order. The byte at 0xC000 is the upper left corner of the screen, the byte at 0xC001 is the next character to the right, and so on. The number of bytes per line is set by the base resolution of the screen, but is generally 80. When a border is displayed, while that limits the number of characters displayed, the layout in memory remains the same.

The text screen has two core resolutions, tied to the refresh rate of the screen: 80 by 60 at 60 Hz, and 80 by 50 at 70 Hz. Beyond that, the character display my be made double width or double height, or both. This gives the following possible character displays: $80 \times 60$, $40 \times 60$, $80 \times 30$, $40 \times 30$, $80 \times 50$, $40 \times 50$, $80 \times 25$, and $40 \times 25$.

\example{Print an A to the Screen}
\begin{verbatim}
    lda $0001       ; Save the current MMU setting
    pha

    lda #$02        ; Swap I/O Page 2 into bank 6
    sta $0001

    lda #’A’        ; Write ‘A’ to the upper left corner
    sta $C000

    pla             ; Restore the old MMU setting
    sta $0001
\end{verbatim}

Note: this example does not set the font or the color, so depending on how your \jr\ is initialized, you may not see an actual ``A'' on the screen.

\section{Text Color LUTs}

Characters in TinyVicky text mode have two colors: the foreground and the background. The foreground and background colors are picked for each character out of two different palettes of 16 colors each. The colors in the palettes are picked from the full range of colors \jr\ can produce, which is more than 16 million colors. This is all managed through two color lookup tables (LUTs) provided by TinyVicky: a text foreground color LUT, and a text background color LUT.

The text LUTs are stored in I/O page 0. The foreground LUT starts at 0xD800, and the background LUT starts at 0xD840.

Each LUT is a list of 16 entries. Each entry is a set of four bytes: blue, green, red, and alpha. Each byte indicates how much of that primary color is present as a component of the actual color. The values range from 0 (none) to 255 (as much as possible). Currently, the alpha channel is not used and is there for future expansion.

\begin{table}[h]
    \begin{center}
        \begin{tabular}{|c|c|c|c|c|c|c|c|} \hline
            Index & R/W & Foreground & Background & 0 & 1 & 2 & 3 \\ \hline\hline
            0 & W & \verb+0xD800+ & \verb+0xD840+ & BLUE\_0 & GREEN\_0 & RED\_0 & X \\ \hline
            1 & W & \verb+0xD804+ & \verb+0xD844+ & BLUE\_1 & GREEN\_1 & RED\_1 & X \\ \hline
            2 & W & \verb+0xD808+ & \verb+0xD848+ & BLUE\_2 & GREEN\_2 & RED\_2 & X \\ \hline
            3 & W & \verb+0xD80C+ & \verb+0xD84C+ & BLUE\_3 & GREEN\_3 & RED\_3 & X \\ \hline
            4 & W & \verb+0xD810+ & \verb+0xD850+ & BLUE\_4 & GREEN\_4 & RED\_4 & X \\ \hline
            5 & W & \verb+0xD814+ & \verb+0xD854+ & BLUE\_5 & GREEN\_5 & RED\_5 & X \\ \hline
            6 & W & \verb+0xD818+ & \verb+0xD858+ & BLUE\_6 & GREEN\_6 & RED\_6 & X \\ \hline
            7 & W & \verb+0xD81C+ & \verb+0xD85C+ & BLUE\_7 & GREEN\_7 & RED\_7 & X \\ \hline
            8 & W & \verb+0xD820+ & \verb+0xD860+ & BLUE\_8 & GREEN\_8 & RED\_8 & X \\ \hline
            9 & W & \verb+0xD824+ & \verb+0xD864+ & BLUE\_9 & GREEN\_9 & RED\_9 & X \\ \hline
            10 & W & \verb+0xD828+ & \verb+0xD868+ & BLUE\_10 & GREEN\_10 & RED\_10 & X \\ \hline
            11 & W & \verb+0xD82C+ & \verb+0xD86C+ & BLUE\_11 & GREEN\_11 & RED\_11 & X \\ \hline
            12 & W & \verb+0xD830+ & \verb+0xD870+ & BLUE\_12 & GREEN\_12 & RED\_12 & X \\ \hline
            13 & W & \verb+0xD834+ & \verb+0xD874+ & BLUE\_13 & GREEN\_13 & RED\_13 & X \\ \hline
            14 & W & \verb+0xD838+ & \verb+0xD878+ & BLUE\_14 & GREEN\_14 & RED\_14 & X \\ \hline
            15 & W & \verb+0xD83C+ & \verb+0xD87C+ & BLUE\_15 & GREEN\_15 & RED\_15 & X \\ \hline
        \end{tabular}
    \end{center}
    \caption{Text Color Lookup Tables}
    \label{tab:text_luts}
\end{table}

\section{Color Matrix}

The way that text color is selected for each character is through the color matrix. This section of memory is in I/O page 3 and starts at 0xC000 when page 3 is swapped into the CPU’s address space. The layout is precisely the same as the text matrix (e.g. the character at 0xC123 in the text matrix has its color information at 0xC123 in the color matrix).

Each byte in the color matrix specifies two colors by providing an index into each of the two text LUTs. The most significant four bits is the number of the foreground color to use. The number of the least significant four bits is the number of the background color to use.

Let’s say the color value at 0xC123 is 0x45. This means that the foreground color of the character is color 4 from the text foreground LUT, which starts at 0xD810 (0xD800 + 4 * 4), and the background color of the character is 5 from the text background LUT, which starts at 0xD854 (0xD840 + 4 * 5). If the bytes at 0xD810 are 0x00, 0x80, 0x80, that means the foreground will be a medium yellow. If the bytes at 0xD854 are 0xFF, 0x00, 0x00, that means the background will be blue.

\example{Make That ``A'' Yellow on Blue}
\begin{verbatim}
    lda $0001       ; Save the MMU state
    pha

    stz $0001       ; Switch in I/O Page #0

    stz $D810       ; Set foreground #4 to medium yellow
    lda #$80
    sta $D811
    sta $D812

    lda #$FF        ; Set background #5 to blue
    sta $D854
    stz $D855
    stz $D856

    lda #$03        ; Switch to I/O page #3 (color matrix)
    sta $0001

    lda #$45        ; Color will be foreground=4, background=5
    sta $C000

    pla             ; Restore the MMU state
    sta $0001
\end{verbatim}

\section{Entering Text Mode}

Whether or not text mode is being displayed (and in what resolution) is controlled by the VICKY Master Control Registers (see table~\ref{tab:vky_master_ctrl_reg}). For now, we're going to ignore most of the bits, which are used by other display modes. For text mode, we really only care about the TEXT bit, which needs to be set to turn on the text display. The resolution is controlled by DBL\_Y, DBL\_X, and CLK\_70. If we set 0xD000 to 0x01 and 0xD001 to 0x00, that will put us into text mode at $80 \times 60$.

\begin{table}[h]
    \begin{center}
        \begin{tabular}{|c|c|c|c|c|c|c|c|c|c|} \hline
            Address & R/W & 7 & 6 & 5 & 4 & 3 & 2 & 1 & 0 \\ \hline\hline
            \verb+0xD000+ & R/W & X & GAMMA & SPRITE & TILE & BITMAP & GRAPH & OVRLY & TEXT \\ \hline
            \verb+0xD001+ & R/W & \multicolumn{5}{|c|}{X} & DBL\_Y & DBL\_X & CLK\_70 \\ \hline
        \end{tabular}
    \end{center}
    \caption{VICKY Master Control Registers}
    \label{tab:vky_master_ctrl_reg}
\end{table}

\begin{description}
    \item[TEXT] if set (1), text mode display is enabled

    \item[OVRLY] if set, text will be overlayed on graphics

    \item[GRAPH] if set, one or more of the graphics modes may be used

    \item[BITMAP] if set (and GRAPHICS is set), bitmap graphics may be displayed

    \item[TILE] if set (and GRAPHICS is set), tile graphics may be displayed

    \item[SPRITE] if set (and GRAPHICS is set), sprite graphics may be displayed

    \item[GAMMA] if set, gamma correction is enabled

    \item[CLK\_70] if set, the video refresh will be set to 70 Hz mode (640x400 text resolution, 320x200 graphics). If clear,
        the video refresh will be set to 60 Hz (640x480 text resolution, 320x240 graphics).

    \item[DBL\_X] if set, text mode characters will be twice as wide (320 pixels)

    \item[DBL\_Y] if set, text mode characters will be twice as high (240 or 200 pixels, depending on CLK\_70)
\end{description}

\section{Text Fonts}

Character shapes (or ``glyphs,'' if you prefer) are defined in font memory, which is in I/O page 1 and starts at 0xC000. The \jr\ treats each character as a square of pixels, 8 pixels on a side. A pixel may be either in the foreground color for the character or in the background color for the character. The way this is managed is that each character has a sequence of eight bytes in the font memory. Each byte represents a row in the character, and each bit represents a pixel in the row ($\blacksquare$ for foreground, $\square$ for background).

As an example, let's say we wanted to have a fancy ``F'' for character 0:

\begin{table}[h]
    \begin{center}
        \begin{tabular}{|c|c|c|c|c|c|c|c|c|} \hline
            $\square$ & $\square$ & $\square$ & $\blacksquare$ & $\blacksquare$ & $\blacksquare$ & $\blacksquare$ & $\blacksquare$ & 0x1F \\ \hline
            $\square$ & $\square$ & $\blacksquare$ & $\blacksquare$ & $\square$ & $\square$ & $\square$ & $\square$ & 0x30 \\ \hline
            $\square$ & $\square$ & $\blacksquare$ & $\blacksquare$ & $\square$ & $\square$ & $\square$ & $\square$ & 0x30 \\ \hline
            $\square$ & $\blacksquare$ & $\blacksquare$ & $\blacksquare$ & $\blacksquare$ & $\blacksquare$ & $\square$ & $\square$ & 0x7C \\ \hline
            $\square$ & $\blacksquare$ & $\blacksquare$ & $\square$ & $\square$ & $\square$ & $\square$ & $\square$ & 0x60 \\ \hline
            $\blacksquare$ & $\blacksquare$ & $\square$ & $\square$ & $\square$ & $\square$ & $\square$ & $\square$ & 0xC0 \\ \hline
            $\blacksquare$ & $\blacksquare$ & $\square$ & $\square$ & $\square$ & $\square$ & $\square$ & $\square$ & 0xC0 \\ \hline
        \end{tabular}
    \end{center}
    \caption{A sample character}
    \label{tab:text_font}
\end{table}

The glyph to display would be defined by the eight byte sequence 0x1F, 0x30, 0x30, 0x7C, 0x60, 0xC0, 0xC0. We would store that sequence in I/O page 0, starting at 0xC000 (0x1F), through 0xC007 (0xC0). After that was set, any time the byte 0x00 is written to screen memory, the glyph ``F'' would be displayed in that position.

\section{Text Cursor}

\jr\ has a text mode cursor. The text mode cursor is implemented as a character which is displayed in a $(x, y)$ position on the screen, visually replacing the character ordinarily at that position. It may be displayed continuously, or it may flash at one of four rates. When flashing, that position in the text screen will alternate between the text cursor and the character at that position in the text matrix. The color for the text cursor comes from the color for the position on the screen as specified in the color matrix. In other words, the text cursor does not have its own color.

\begin{table}[h]
    \begin{center}
        \begin{tabular}{|c|c|c|c|c|c|c|c|c|c|c|} \hline
            Address & R/W & Name & 7 & 6 & 5 & 4 & 3 & 2 & 1 & 0 \\\hline\hline
            \verb+0xD010+ & R/W & CCR & \multicolumn{2}{|c|}{---} & FLASH\_EN & \multicolumn{2}{|c|}{RATE} & ENABLE \\ \hline
            \verb+0xDC12+ & R/W & CCH & \multicolumn{8}{|c|}{Cursor character} \\ \hline
            \verb+0xDC14+ & R/W & \multirow{2}{*}{CURX} & X7 & X6 & X5 & X4 & X3 & X2 & X1 & X0 \\ \hline
            \verb+0xDC15+ & R/W &  & X15 & X14 & X13 & X12 & X11 & X10 & X9 & X8 \\ \hline
            \verb+0xDC16+ & R/W & \multirow{2}{*}{CURY} & Y7 & Y6 & Y5 & Y4 & Y3 & Y2 & Y1 & Y0 \\ \hline
            \verb+0xDC17+ & R/W &  & Y15 & Y14 & Y13 & Y12 & Y11 & Y10 & Y9 & Y8 \\ \hline
        \end{tabular}
    \end{center}
    \caption{Text Cursor Registers}
    \label{tab:txt_crsr_reg}
\end{table}

\begin{description}
    \item[ENABLE] if this flag is set (1), the cursor is enabled

    \item[FLASH\_EN] if this flag is set (1), the cursor will flash. If clear (0), it will just be steady

    \item[RATE] these two bits set the rate at which the cursor flashes (see table:~\label{tab:txt_crsr_rates})

    \item[CCH] the character code for the cursor character to display

    \item[CURX] the column number (16-bit) for the cursor

    \item[CURY] the row number (16-bit) for the cursor
\end{description}

\begin{table}[h]
    \begin{center}
        \begin{tabular}{|c|c|c|} \hline
            RATE1 & RATE0 & Rate \\\hline\hline
            0 & 0 & 1s \\ \hline
            0 & 1 & $1/2$s \\ \hline
            1 & 0 & $1/4$s \\ \hline
            1 & 1 & $1/5$s \\ \hline
        \end{tabular}
    \end{center}
    \caption{Text Cursor Flash Rates}
    \label{tab:txt_crsr_rates}
\end{table}
