\documentclass[oneside]{book}
\usepackage{concmath}
\usepackage[OT1]{fontenc}
\usepackage[a4paper, margin=1in]{geometry}
\usepackage{multirow}
\newcommand{\jr}{{\sc C256jr}}
\newcommand{\microsec}{$\mu$s}
\begin{document}

\title{Foenix \jr\ User's Manual}
\author{Peter Weingartner}
\maketitle

\chapter{Introduction}

Put the introduction here.

\chapter{\jr\ Basics}

\chapter{Memory Management}

The \jr\ has 256 KB of system RAM which can be used for programs, data, and graphics. It also has 512 KB of read-only flash memory that can be used by whatever operating system is installed. Now, the 65C02 CPU at the heart of the \jr\ has an address space of only 64 KB, so how can it access all this memory, not to mention the I/O devices on the system? The answer is paging. The \jr\ has a special memory management unit (MMU) that can swap banks of memory or I/O registers into and out of the memory space of the CPU.

To understand how it all works, we first need to look at how RAM and flash memory are handled by the \jr. Because there are 768 KB of total storage on the system, the system has a 20-bit address bus to manage the memory. RAM and flash have address on that 20-bit bus as shown in table~\ref{tab:memory}.

\begin{table}
	\begin{center}
		\begin{tabular}{| l | l | l |} \hline
			Start & End & Memory Type \\ \hline\hline
		  	{\tt 0x00000} & {\tt 0x2FFFF} & System RAM (256 KB)\\ \hline
			{\tt 0x30000} & {\tt 0x7FFFF} & Reserved for future use (256 KB)\\ \hline
		  	{\tt 0x80000} & {\tt 0xFFFFF} & Flash Memory (512 KB) \\ \hline
		\end{tabular}
	\end{center}
	\caption{C256jr memory layout}
	\label{tab:memory}
\end{table}

This memory is divided up into ``banks'' of 8 KB each. The 16-bit address space of the CPU is also divided up into 8 KB banks. The MMU allows the program to assign any bank of system memory to any bank of the CPU's memory. It does this through the use of memory look-up tables (LUT), which provide the upper bits needed to select the bank out of system memory for any given bank in CPU memory. It takes 13 bits to specify an address within 8 KB, which means for a 16-bit address from the CPU, the upper 3 bits are the bank number. Since the system bus is 20 bits, a bank number there is 7 bits. So a LUT must provide a 7-bit system bank number for each 3-bit bank number provided by the CPU.

The \jr's MMU supports up to four LUTs, only one of which is active at any given moment. This allows programs to define four different memory layouts and switch between them quickly, without having to alter a LUT on the fly.

\begin{table}
	\begin{center}
		\begin{tabular}{| c | c || c | c | c | c | c | c | c | c |} \hline
			Bank & A[15..13] & Start & End \\ \hline\hline
			0 & 000 & 0x0000 & 0x1FFF \\ \hline
			1 & 001 & 0x2000 & 0x3FFF \\ \hline
			2 & 010 & 0x4000 & 0x5FFF \\ \hline
			3 & 011 & 0x6000 & 0x7FFF \\ \hline
			4 & 100 & 0x8000 & 0x9FFF \\ \hline
			5 & 101 & 0xA000 & 0xBFFF \\ \hline
			6 & 110 & 0xC000 & 0xDFFF \\ \hline
			7 & 111 & 0xE000 & 0xFFFF \\ \hline
		\end{tabular}
	\end{center}
	\caption{CPU Memory Banks}
	\label{tab:mem_banks}
\end{table}

Of the eight CPU memory banks, one is special. Bank 6 can be mapped to memory as the rest can, or it can be mapped to I/O registers, which are not memory mapped in the same way as RAM and flash. All I/O devices on the \jr\ therefore live within 0xC000 through 0xDFFF on the CPU, but only if the MMU is set to map I/O to bank 6. There is quite a lot of I/O to access on the \jr, so there are four different banks of I/O registers and memory that can be mapped to bank 6 (see table~\ref{tab:io_banks}).

\begin{table}
	\begin{center}
		\begin{tabular}{| l | l |} \hline
			I/O Bank & Purpose \\ \hline\hline
			0 & Low level I/O registers \\ \hline
			1 & Text display font memory and graphics color LUTs \\ \hline
			2 & Text display character matrix \\ \hline
			3 & Text display color matrix \\ \hline
		\end{tabular}
	\end{center}
	\caption{I/O Banks}
	\label{tab:io_banks}
\end{table}

\begin{table}
	\begin{center}
		\begin{tabular}{| c | c || c | c | c | c | c | c | c | c |} \hline
			Address & Name & 7 & 6 & 5 & 4 & 3 & 2 & 1 & 0 \\ \hline\hline
			0x0000 & SYS\_CTRL\_0 & EDIT\_EN & RSVD & \multicolumn{2}{|c|}{EDIT\_LUT} & RSVD & RSVD & \multicolumn{2}{|c|}{ACT\_LUT} \\ \hline
			0x0001 & SYS\_CTRL\_1 & \multicolumn{5}{|c|}{RSVD} & IO\_DISABLE & \multicolumn{2}{|c|}{IO\_PAGE} \\ \hline
		\end{tabular}
	\end{center}
	\caption{MMU Registers}
	\label{tab:mmu_registers}
\end{table}

The MMU is controlled through two main registers, which are always at locations 0x0000 and 0x0001 in the CPU's address space (see table~\ref{tab:mmu_registers}). These registers allow programs to select an active LUT, edit a LUT, and control bank 6:

\begin{itemize}
	\item[ACT\_LUT] these two bits specify which LUT (0 - 3) is used to translate CPU bus address to system bus addresses.

	\item[EDIT\_EN] if set (1), this bit allows a LUT to be edited by the program, and memory addresses 0x0010 - 0x0017 will be used by the LUT being edited. If clear (0), those memory locations will be standard memory locations and will be mapped like the rest of bank 0.

	\item[EDIT\_LUT] if EDIT\_EN is set, these two bits will specify which LUT (0 - 3) is being editted and will appear in memory addresses 0x0010 - 0x0017.

	\item[IO\_DISABLE] if set (1), bank 6 is mapped like any other memory bank. If clear (0), bank 6 is mapped to I/O memory.

	\item[IO\_PAGE] if IO\_DISABLE is clear, these two bits specify which bank of I/O memory (0 - 3) is mapped to bank 6.
\end{itemize}

\section{Example: Setting up a LUT}

In this example, we will set up LUT 1 so that the first six banks of CPU memory map to the first banks of RAM, bank 7 of CPU memory maps to the first bank of flash memory, and bank 6 maps to the first I/O bank.

\begin{verbatim}
    lda #$90      ; Active LUT = 0, Edit LUT#1
    sta $0000

    ldx #0        ; Start at bank 0
l1: txa           ; First 6 banks will just be the first banks of RAM
    sta $0010,x   ; Set the LUT mapping for this bank
    inx           ; Move to the next bank
    cpx #6        ; Until we get to bank 6
    bne l1

    lda #$40      ; Bank 7 maps to $80000, first bank of flash
    sta $0017

    stz $0001     ; Bank 6 should be I/O bank 0

    lda #$01      ; Turn off LUT editting, and switch to LUT#1
    sta $0000
\end{verbatim}

\chapter{The Display}

\section{Using the Text Screen}

\section{Bitmapped Graphics}

\section{Sprites}

\section{Tiles}

\chapter{Sound}

The \jr\ line has a couple of sound chips, which chips are present depends upon the model. All of the machines have built-in the SN76489 (called the ``PSG'' here), which was used by many vintage machines including the TI99/4A, the BBC Micro, the IBM PCjr, and the Tandy 1000. The PSG chips on the \jr\ are actually implemented as part of the TinyVicky FPGA. The \fjr\ also has two sockets on the board that may be populated with SID chips (either the original 6581, the later 8580, or any of the new FPGA replacements). The \fk\ implements these SID chips in the FPGA but also includes a physical OPL3 chip on the board.

\section*{CODEC}

The \jr\ (and indeed all the Foenix computers up to this point) makes use of a WM8776 CODEC chip. You can think of the CODEC as the central switchboard for audio on the \jr. The CODEC chip has inputs for several audio channels (both analog and digital), and each audio device on the \jr\ is routed to an input on the CODEC. The CODEC then has outputs for audio line level and headphones. The CODEC will convert analog inputs to digital, mix all the audio inputs according to its settings, and then convert the resulting digital audio to analog and drive the outputs. With the CODEC, you can turn on and off the various input channels, control the volume, and mute or enable the different outputs.

The CODEC is a rather complex chip with many features, and the full details are really beyond the scope of this document. Most programs for the \jr\ will not need to use it or will only use it in very specific ways. Therefore, this document will really just show how to access it and initialize it and then leave a reference to the data sheet for the chip that has the complete data on the chip.

Raw access to the CODEC chip is fairly complex. Fortunately, the FPGA on the \jr\ provides three registers to simply access for programs. The FPGA takes care of the actual timing of transmitting data to the CODEC, serializing the data correctly, and so on. All the program needs to know about are the correct format for the 16-bit command words that are sent to the CODEC, and then a status register to monitor.

The CODEC commands are based around a number of registers. Each command is really just writing values to those registers. The command words are 16-bits wide, with the 7 most significant bits being the number of the register to write, and the 9 least significant bits being the data to write. For instance, there is a register to enable and disable the headphone output. Bit 0 of the register controls whether or not the headphone output is enabled (0 = enabled, 1 = disabled). The register number is 13. So, to disable the output on the headphones, we would need to write \verb+000000001+ to register 13. The register number in binary is \verb+0001101+, So the command word we would need to send is \verb+0001101000000001+ or \verb+0x1A01+.

The registers for the CODEC on the \jr\ are shown in table~\ref{tab:codec_registers}.

\begin{table}[ht]
	\begin{center}
		\begin{tabular}{|c|c|c|c|c|c|c|c|c|c|l|} \hline
			Address & R/W & 7 & 6 & 5 & 4 & 3 & 2 & 1 & 0 & Purpose \\ \hline \hline
			\verb+0xD620+ & W & D7 & D6 & D5 & D4 & D3 & D2 & D1 & D0 & Command Low \\ \hline
			\verb+0xD621+ & W & R6 & R5 & R4 & R3 & R2 & R1 & R0 & D8 & Command High \\ \hline
			\verb+0xD622+ & R & \multicolumn{7}{|c|}{X} & BUSY & Status \\ \hline
			\verb+0xD622+ & W & \multicolumn{7}{|c|}{X} & START & Control \\ \hline
		\end{tabular}
	\end{center}
	\caption{CODEC Control Registers}
	\label{tab:codec_registers}
\end{table}

Bit 0 of the status/control register both triggers sending the command (on a write) and indicates if the CODEC is busy receiving a command (writing a 1 triggers the sending of the command, reading a 1 indicates that the CODEC is busy).

So to mute the headphones, we would issue the following:

\begin{verbatim}
wait:  lda $D622    ; Wait for the CODEC to be ready
       and #$01
       cmp #$01
       beq wait     ; Bit 0 = 1, CODEC is still busy... keep waiting

       lda #$01     ; Set command to %0001101000000001, or R13 <- 000000001
       sta $D620
       lda #$1A
       sta $D621

       lda #$01     ; Trigger the transmission of the command to the CODEC
       sta $D622
\end{verbatim}

\section*{Using the PSGs}

The \jr\ has support for dual SN76489 (PSG) sound chips, emulated in the FPGA. The SN76489 was used in several vintage machines, including the TI-99/4A, BBC Micro, IBM PCjr, and Tandy 1000. The chip provides three independent square-wave tone generators and a single noise generator. Each tone generator can produce tones of several frequencies in 16 different volume levels. The noise generator can produce two different types of noise in three different tones at 16 different volume levels.

Access to each PSG is through a single memory address, but that single address allows the CPU to write a value to eight different internal registers. For each tone generator, there is a ten bit frequency (which takes two bytes to set), and a four bit ``attenuation'' or volume level. For the noise generator, there is a noise control register and a noise attenuation register.

\begin{table}[ht]
	\begin{center}
		\begin{tabular}{|c|c|c|c|l|} \hline
			R2 & R1 & R0 & Channel & Purpose \\ \hline \hline
			0 & 0 & 0 & Tone 1 & Frequency \\ \hline
			0 & 0 & 1 & Tone 1 & Attenuation \\ \hline
			0 & 1 & 0 & Tone 2 & Frequency \\ \hline
			0 & 1 & 1 & Tone 2 & Attenuation \\ \hline
			1 & 0 & 0 & Tone 3 & Frequency \\ \hline
			1 & 0 & 1 & Tone 3 & Attenuation \\ \hline
			1 & 1 & 0 & Noise & Control \\ \hline
			1 & 1 & 1 & Noise & Attenuation \\ \hline
		\end{tabular}
	\end{center}
	\caption{SN76489 Channel Registers}
	\label{tab:psg_registers}
\end{table}

There are four basic formats of bytes that can be written to the port, as shown in table~\ref{tab:psg_commands}.

\begin{table}[ht]
	\begin{center}
		\begin{tabular}{|c|c|c|c|c|c|c|c|l|} \hline
			D7 & D6 & D5 & D4 & D3 & D2 & D1 & D0 & Purpose \\ \hline \hline
			1 & R2 & R1 & R0 & F3 & F2 & F1 & F0 & Set the low four bits of the frequency \\ \hline
			0 & X & F9 & F8 & F7 & F6 & F5 & F4 & Set the high six bits of the frequency \\ \hline
			1 & 1 & 1 & 0 & X & FB & F1 & F0 & Set the type and frequency of the noise generator \\ \hline
			1 & R2 & R1 & R0 & A3 & A2 & A1 & A0 & Set the attenuation (four bits) \\ \hline
		\end{tabular}
	\end{center}
	\caption{SN76489 Command Formats}
	\label{tab:psg_commands}
\end{table}

Note: there is a PSG sound device for the left stereo channel and one for the right. The left channel PSG can be accessed at 0xD600, and the right channel at 0xD610. Both are in I/O page 0. There is also a sound ``device'' for managing the left and right PSGs together, which starts at 0xD608. The combined registers work in the same way as the left and right PSGs. Writing to the combined registers is equivalent to writing to the left and right channel registers simultaneously.

The PSGs can be used with their outputs mixed in one of two modes. They can either be used as independent 4 voice stereo sound (one PSG on the left and one on the right), or they can be used as 8 voice monaural sound (both PSGs are routed to both left and right sound channels). This is controlled by the PSG\_ST flag in the system control registers (see page~\pageref{tab:sys_ctrl_reg}).

\subsection*{Attenuation}

All the channels support attenuation or volume control. The PSG expresses the loudness of the sound with how much it is attenuated or dampened. Therefore, an attenuation of 0 will be the loudest sound, while an attenuation of 15 will make the channel silent.

\subsection*{Tones}

Each of the three sound channels generates simple square waves. The frequency generated depends upon the system clock driving the chip and the number provided in the frequency register. The relationship is:
\[
f = \frac{C}{32n}
\]
where $f$ is the frequency produced, $C$ is the system clock, and $n$ is the number provided in the register. Expressed a different way, the value we need to produce a given frequency can be computed as:
\[
n = \frac{C}{32f}
\]
For the \jr\ the system clock is 3.57 MHz, which means:
\[
n = \frac{111,563}{f}
\]
So, let us say we want channel 1 to produce a concert A, which is 440Hz at maximum volume. The value we need to set for the frequency code is $111,320 / 440 = 253$ or 0xFE. We can do that with this code:

\begin{verbatim}
    lda #$90        ; %10010000 = Channel 1 attenuation = 0
    sta $D600       ; Send it to left PSG
    sta $D610       ; Send it to right PSG

    lda #$8E        ; %10001100 = Set the low 4 bits of the frequency code
    sta $D600       ; Send it to left PSG
    sta $D610       ; Send it to right PSG

    lda #$0F        ; %00001111 = Set the high 6 bits of the frequency
    sta $D600       ; Send it to left PSG
    sta $D610       ; Send it to right PSG
\end{verbatim}
To turn it off later, we just need to write:

\begin{verbatim}
    lda #$9F        ; %10011111 = Channel 1 attenuation = 15 (silence)
    sta $D600       ; Send it to left PSG
    sta $D610       ; Send it to right PSG
\end{verbatim}

\subsection*{Noise}

Noise works differently from tones, since it is random. The noise generator on the PSG can produce two styles of noise determined by the FB bit: white noise (FB = 1), and periodic (FB = 0). The noise has a sort of frequency, based on either the system clock or the current output of tone 3. This frequency is set using the F1 and F0 bits:

\begin{table}[ht]
	\begin{center}
		\begin{tabular}{|c|c|c|c|l|} \hline
			F1 & F0 & Frequency \\ \hline \hline
			0 & 0 & $C / 512$ \\ \hline
			0 & 1 & $C / 1024$ \\ \hline
			1 & 0 & $C / 2048$ \\ \hline
			1 & 1 & Tone 3 output \\ \hline
		\end{tabular}
	\end{center}
	\caption{SN76489 Noise Frequencies}
	\label{tab:psg_noise_freq}
\end{table}

As an example, to set white noise of the highest frequency ($C / 512$ or around 6 kHz), we could use the code:

\begin{verbatim}
    lda #$F0        ; %10010000 = Channel 3 attenuation = 0
    sta $D600       ; Send it to left PSG
    sta $D610       ; Send it to right PSG

    lda #$E4        ; %11100100 = white noise, f = C/512
    sta $D600       ; Send it to left PSG
    sta $D610       ; Send it to right PSG
\end{verbatim}
To turn it off later, we just need to write:

\begin{verbatim}
    lda #$FF        ; %1ff11111 = Channel 3 attenuation = 15 (silence)
    sta $D600       ; Send it to left PSG
    sta $D610       ; Send it to right PSG
\end{verbatim}

\section*{Using the SIDs}

The SID is a full-featured analog sound synthesizer, and a full explanation of how to use it is really beyond the scope of this document. In this document, I will provide just an introduction to the chip and list the register addresses for the SID chips (see table~\ref{tab:sid_registers}). For the \fjr, the SID chips are optional. The board comes with two unpopulated sockets into which either genuine SID chips or the various FPGA replacements may be installed. For the \fk, there is no socket for SIDs, but the two SID chips are provided by the built-in FPGAs.

The SID chip provides three independent voices (so it can play three notes at once). The three voices are almost identical in their features, with voice 3 being the only one different. Each voice can produce one of four basic sound wave forms: randomized noise, square waves, saw tooth waves, and triangle waves. These waves can be generated over a range of frequencies, and for the square waves, the width of the pulse ({\it i.e.} duty cycle) may be adjusted.

The type of wave form produced by a voice is controlled by the NOISE, PULSE, SAW, and TRI bits. If NOISE is set to 1, the output is random noise. If PULSE is set, a square wave is produced. If SAW is set, a saw tooth wave is produced. If TRI is set, the voice produces a triangle wave. If PULSE is set, the duty cycle of the square wave (or pulse width, if you prefer) is set by the PW bits according to the formula
${\rm PW} / 40.95$ (expressed as a percent).

\begin{table}[ht]
	\begin{center}
		\begin{tabular}{|c|c|c|c|c|c|c|c|c|c|c|} \hline
			Voice & Offset & R/W & 7 & 6 & 5 & 4 & 3 & 2 & 1 & 0 \\ \hline\hline
            \multirow{7}{*}{V1} & 0 & W & F7 & F6 & F5 & F4 & F3 & F2 & F1 & F0 \\ \cline{2-11}
            & 1 & W & F15 & F14 & F13 & F12 & F11 & F10 & F9 & F8 \\ \cline{2-11}
            & 2 & W & PW7 & PW6 & PW5 & PW4 & PW3 & PW2 & PW1 & PW0 \\ \cline{2-11}
            & 3 & W & \multicolumn{4}{|c|}{X} & PW11 & PW10 & PW9 & PW8 \\ \cline{2-11}
            & 4 & W & NOISE & PULSE & SAW & TRI & TEST & RING & SYNC & GATE \\ \cline{2-11}
            & 5 & W & ATK3 & ATK2 & ATK1 & ATK0 & DLY3 & DLY2 & DLY1 & DLY0 \\ \cline{2-11}
            & 6 & W & STN3 & STN2 & STN1 & STN0 & RLS3 & RLS2 & RLS1 & RLS0 \\ \hline\hline

            \multirow{7}{*}{V2} & 7 & W & F7 & F6 & F5 & F4 & F3 & F2 & F1 & F0\\ \cline{2-11}
            & 8 & W & F15 & F14 & F13 & F12 & F11 & F10 & F9 & F8 \\ \cline{2-11}
            & 9 & W & PW7 & PW6 & PW5 & PW4 & PW3 & PW2 & PW1 & PW0 \\ \cline{2-11}
            & 10 & W & \multicolumn{4}{|c|}{X} & PW11 & PW10 & PW9 & PW8 \\ \cline{2-11}
            & 11 & W & NOISE & PULSE & SAW & TRI & TEST & RING & SYNC & GATE \\ \cline{2-11}
            & 12 & W & ATK3 & ATK2 & ATK1 & ATK0 & DLY3 & DLY2 & DLY1 & DLY0 \\ \cline{2-11}
            & 13 & W & STN3 & STN2 & STN1 & STN0 & RLS3 & RLS2 & RLS1 & RLS0 \\ \hline\hline
		\end{tabular}
	\end{center}
	\caption{SID V1 and V2 Registers}
	\label{tab:sid_registers}
\end{table}

\begin{table}[ht]
	\begin{center}
		\begin{tabular}{|c|c|c|c|c|c|c|c|c|c|c|} \hline
			Voice & Offset & R/W & 7 & 6 & 5 & 4 & 3 & 2 & 1 & 0 \\ \hline\hline
            \multirow{7}{*}{V3} & 14 & W & F7 & F6 & F5 & F4 & F3 & F2 & F1 & F0 \\ \cline{2-11}
            & 15 & W & F15 & F14 & F13 & F12 & F11 & F10 & F9 & F8 \\ \cline{2-11}
            & 16 & W & PW7 & PW6 & PW5 & PW4 & PW3 & PW2 & PW1 & PW0 \\ \cline{2-11}
            & 17 & W & \multicolumn{4}{|c|}{X} & PW11 & PW10 & PW9 & PW8 \\ \cline{2-11}
            & 18 & W & NOISE & PULSE & SAW & TRI & TEST & RING & SYNC & GATE \\ \cline{2-11}
            & 19 & W & ATK3 & ATK2 & ATK1 & ATK0 & DLY3 & DLY2 & DLY1 & DLY0 \\ \cline{2-11}
            & 20 & W & STN3 & STN2 & STN1 & STN0 & RLS3 & RLS2 & RLS1 & RLS0 \\ \hline\hline

            \multirow{4}{*}{} & 21 & W & \multicolumn{5}{|c|}{X} & FC2 & FC1 & FC0 \\ \cline{2-11}
            & 22 & W & FC10 & FC9 & FC8 & FC7 & FC6 & FC5 & FC4 & FC3 \\ \cline{2-11}
            & 23 & W & RES3 & RES2 & RES1 & RES0 & EXT & FILTV3 & FILTV2 & FILTV1 \\ \cline{2-11}
            & 24 & W & MUTEV3 & HIGH & BAND & LOW & VOL3 & VOL2 & VOL1 & VOL0 \\ \hline
		\end{tabular}
	\end{center}
	\caption{SID V3 and Miscellaneous Registers}
	\label{tab:sid_registers_v3_global}
\end{table}

The frequency of the waveform is set by the bits \verb+F[15..0]+. This number sets the actual frequency according the the formula:

\[
f_{\rm out} = \frac{FC}{16777216}
\]
where: $f_{\rm out}$ is the output frequency, $F$ is the number set in the registers, and $C$ is the system clock driving the SIDs. For the \jr, $C$ is 1.022714 MHz, so the formula for the \jr\ is:

\[
f_{\rm out} = \frac{F}{16.405}
\]
or:
\[
F = 16.404f_{\rm out}
\]
For example: concert A, which is 440 Hz, would be: $F = 16.405 \times 440 \approx 7218$. So, to play a concert A, you would set the frequency to 7218, or 0x1C32.

Each of the three voices has a sound ``envelope'' which changes the volume of the sound during the duration of the note. There are four phases to the sound envelope: attack, decay, sustain, and release (ADSR). When the note first starts playing (that is, the GATE bit for the voice is set to 1), it starts at the attack phase when the volume starts at zero and goes up to the current maximum volume (which is controlled by VOL3-0). How fast this happens is determined by the attack rate (ATK3-0 in the registers). Once the volume reaches the maximum, the volume goes down again to the sustain volume. This phase is called decay, and the speed at which the volume drops is determined by the DCY3-0 register values. Next, the envelope enters the sustain phase, where the volume is held steady at the sustain level (STN3-0). It stays here until the note is to stop playing (GATE is set to 0). At this point, the envelope enters the release stage, where the volume drops back to zero at the release rate (RLS3-0).

The ADSR envelope allows the SID chip to mimic the qualities of various musical instruments or shape various sound effects. For instance, a pipe organ's notes are typically either on or off, so the attack, decay, and release rates would be set to be instantaneous, and the sustain level would be set to full. A piano, on the other hand tends to have a sharp, somewhat percussive sound at the beginning with the note holding a long time on release if not dampened.

While the different voices are independent, they can be set to alter one another through two different effects: synchronization, and ring modulation. With these features, the voices can interact with each other in the following pairs:

\begin{itemize}
\item Voice 1 $\rightarrow$ Voice 2
\item Voice 2 $\rightarrow$ Voice 3
\item Voice 3 $\rightarrow$ Voice 1
\end{itemize}

\subsection*{Ring Modulation}

If a voice's RING bit is set and the voice is set to use the triangle wave form (TRI is set), then the triangle wave will be replaced by the combination of the two voice's frequencies. So if the RING bit of voice 1 is set, the result will be the ring modulation of voice 1 and voice 3. Ring modulation tends to produce harmonics and overtones and can be used for bell like sounds.

\subsection*{Synchronization}

If a voice's SYNC bit is set, the frequency it produces will be synchronized to the controlling voice. So if voice 1's SYNC bit is set, its frequency will be synchronized to voice 3.

NOTE: Voice 3 can be muted by setting MUTEV3. This is useful to have the wave forms generated by voice 3 be used for ring modulation and synchronization without having voice 3's wave forms being actually audible.

\subsection*{Filtering}

The SID chip can apply a filter to the audio before sending it out for amplification. The filter works at an adjustable frequency and may be used as either a high-pass filter (if HIGH is set), a low-pass filter (if LOW is set), or as a band-pass filter (if BAND is set). The filter frequency is set by the bits FC0-10. The filter may be applied or not to each voice independently. Bits FILTV1, FILTV2, and FILTV3 control whether the filter is applied to voices 1, 2, and 3 respectively. Finally, a resonance effect may be tuned on the filter using the RES0-3 bits: 0 indicates no resonance, and 15 indicates maximum resonance.

\section*{OPL3}

\begin{note}
	This section is relevant to the \fk\ only. It does not apply to any of the \jr\ revisions.
\end{note}

The \fk\ includes a physical OPL3 sound chip, specifically the Yamaha YMF262 sound chip. The OPL3 provides for complex, FM synthesized sound, which allows numerous oscillators to be combined in various ways to generate musical tones. An explanation of the various registers and functions provided by the OPL3 device is well outside the scope of this manual as it deserves its own book. Only how to access those ports will be covered here.

The OPL3 provides many registers or ports for setting the various parameters. These ports are arranged in an address space of 9 bits (0x000--0x1FF). To access these ports, the CPU must first write the address of the port desired into one of two address registers. It then must write the data to be written to that port into the data register. To maintain compatibility with older versions of the Yamaha FM synthesizer chips, these ports are accessed through two different sets of address registers. For ports 0x000 through 0x0FF, the first address register is used. For ports 0x100 through 0x1FF, the second address register is used. See table~\ref{tab:opl3_registers}.

\begin{table}[ht]
	\begin{center}
		\begin{tabular}{|c|c|l|} \hline
			Address & R/W & Purpose \\ \hline \hline
			\verb+0xD580+ & W & Address pointer register for ports 0x000--0x0FF \\ \hline
			\verb+0xD581+ & W & Data register for all ports \\ \hline
			\verb+0xD582+ & W & Address pointer register for ports 0x100--0x1FF \\ \hline
		\end{tabular}
	\end{center}
	\caption{OPL3 Registers}
	\label{tab:opl3_registers}
\end{table}

\chapter{Interrupt Controller}

\chapter{Tracking Time}

\section*{Interval Timers}

The \jr\ provides two 24-bit timers. The two timers work on different clocks: timer 0 works off the CPU clock (6.29 MHz), while timer 1 works off the start-of-frame timing (either 60 Hz or 70 Hz, depending on the resolution). The timers have a few features in how they time things:

\begin{itemize}
    \item they can count up from 0 or down from a starting value

    \item they can be set to trigger an interrupt on a specific value

    \item they can either reload a start value or reset the value to 0 on reaching the target value
\end{itemize}

\begin{table}[ht]
    \begin{center}
        \begin{tabular}{|c|c|c|c|c|c|c|c|c|c|c|} \hline
            Address & R/W & Name & 7 & 6 & 5 & 4 & 3 & 2 & 1 & 0 \\\hline\hline
			\verb+D650+ & W & T0\_CTR & --- & \multicolumn{3}{|c|}{---} & UP & LD & CLR & EN \\ \hline
			\verb+D650+ & R & T0\_STAT & \multicolumn{7}{|c|}{---} & EQ \\ \hline

			\verb+D651+ & R/W & \multirow{3}{*}{T0\_VAL} & V7 & V6 & V5 & V4 & V3 & V2 & V1 & V0 \\ \cline{1-2}\cline{4-11}
			\verb+D652+ & R/W &  & V15 & V14 & V13 & V12 & V11 & V10 & V9 & V8 \\ \cline{1-2}\cline{4-11}
			\verb+D653+ & R/W &  & V23 & V22 & V21 & V20 & V19 & V18 & V7 & V6 \\ \hline

			\verb+D654+ & R/W & T0\_CMP\_CTR & \multicolumn{6}{|c|}{---} & RELD & RECLR \\ \hline

			\verb+D655+ & R/W & \multirow{3}{*}{T0\_CMP} & C7 & C6 & C5 & C4 & C3 & C2 & C1 & C0 \\ \cline{1-2}\cline{4-11}
			\verb+D656+ & R/W &  & C15 & C14 & C13 & C12 & C11 & C10 & C9 & C8 \\ \cline{1-2}\cline{4-11}
			\verb+D657+ & R/W &  & C23 & C22 & C21 & C20 & C19 & C18 & C17 & C16 \\ \hline\hline

			\verb+D658+ & W & T1\_CTR & INT\_EN & \multicolumn{3}{|c|}{---} & UP & LD & CLR & EN \\ \hline
			\verb+D658+ & R & T1\_STAT & \multicolumn{7}{|c|}{---} & EQ \\ \hline

			\verb+D659+ & R/W & \multirow{3}{*}{T1\_VAL} & V7 & V6 & V5 & V4 & V3 & V2 & V1 & V0 \\ \cline{1-2}\cline{4-11}
			\verb+D65A+ & R/W &  & V15 & V14 & V13 & V12 & V11 & V10 & V9 & V8 \\ \cline{1-2}\cline{4-11}
			\verb+D65B+ & R/W &  & V23 & V22 & V21 & V20 & V19 & V18 & V7 & V6 \\ \hline

			\verb+D65C+ & R/W & T1\_CMP\_CTR & \multicolumn{6}{|c|}{---} & RELD & RECLR \\ \hline

			\verb+D65D+ & R/W & \multirow{3}{*}{T1\_CMP} & C7 & C6 & C5 & C4 & C3 & C2 & C1 & C0 \\ \cline{1-2}\cline{4-11}
			\verb+D65E+ & R/W &  & C15 & C14 & C13 & C12 & C11 & C10 & C9 & C8 \\ \cline{1-2}\cline{4-11}
			\verb+D65F+ & R/W &  & C23 & C22 & C21 & C20 & C19 & C18 & C17 & C16 \\ \hline
        \end{tabular}
    \end{center}
    \caption{Timer Registers}
    \label{tab:timer_reg}
\end{table}

There are five registers for each timer:
\begin{description}
    \item[CTR] the master control register for the timer. There are five flags:
        \begin{description}
            \item[UP] if set, the timer will count up. If clear, it will count down.

            \item[CLR] if set, the timer will reset to 0

            \item[LD] if set, the timer will be set to the last value written to VAL

            \item[EN] if set, the timer will count clock ticks
        \end{description}

    \item[STAT] this register (read on the same address as CTR) has just one flag EQ, which indicates if the timer has reached the target value

    \item[VAL] when read, gives the current value of the timer. When written, sets the value to use when loading the timer.

    \item[CMP\_CTR] this register contains two flags to control what happens when the target value is reached. When RECLR is set, the timer will return to 0 on reaching the target value. When RELD is set, the timer will be set to the last value written to VAL.

    \item[CMP] this register contains the target value for comparison
\end{description}

\section*{Real Time Clock}

For programs needing to keep track of time, \jr\ provides a real time clock chip (RTC), the bq4802. This chip, keeps track of the year (including century), month, day, hour (in 12 or 24 hour mode), minute, and second. The coin cell battery on the \jr\ motherboard is to provide power to the RTC so it can continue tracking time even when the \jr\ is turned off or unplugged. Additionally, the RTC can send interrupts to the CPU, either periodically or at a specific time.

The RTC is relatively straightforward to use, but one potentially tricky thing to keep in mind is that there is a specific procedure to follow when reading or writing the date-time. As well as the registers the CPU can access, the RTC has internal registers which are constantly updating as time progresses. Normally, the internal registers update their external counterparts, but this should not be allowed to happen while the CPU is getting or setting the externally facing registers. So, to access the external registers, the program must first disable the automatic updates to the external registers. Then it can read or write the external registers. Then it can re-enable the automatic updates. If the program has changed the registers, when updates are re-enabled the data in the external registers will be sent to the internal registers in one action. This keeps the time information consistent.

\begin{table}[ht]
	\begin{center}
		\begin{tabular}{| c | c | c || c | c | c | c | c | c | c | c |} \hline
			Address & R/W & Name & 7 & 6 & 5 & 4 & 3 & 2 & 1 & 0 \\ \hline\hline
			\verb+0xD690+ & R/W & Seconds & 0 & \multicolumn{3}{|c|}{second 10s digit} & \multicolumn{4}{|c|}{second 1s digit} \\ \hline
            \verb+0xD691+ & R/W & Seconds Alarm & 0 & \multicolumn{3}{|c|}{second 10s digit} & \multicolumn{4}{|c|}{second 1s digit} \\ \hline
            \verb+0xD692+ & R/W & Minutes & 0 & \multicolumn{3}{|c|}{minute 10s digit} & \multicolumn{4}{|c|}{minute 1s digit} \\ \hline
            \verb+0xD693+ & R/W & Minutes Alarm & 0 & \multicolumn{3}{|c|}{minute 10s digit} & \multicolumn{4}{|c|}{minute 1s digit} \\ \hline
            \verb+0xD694+ & R/W & Hours & AM/PM & 0 & \multicolumn{2}{|c|}{hour 10s digit} & \multicolumn{4}{|c|}{hour 1s digit} \\ \hline
            \verb+0xD695+ & R/W & Hours Alarm & AM/PM & 0 & \multicolumn{2}{|c|}{hour 10s digit} & \multicolumn{4}{|c|}{hour 1s digit} \\ \hline
            \verb+0xD696+ & R/W & Days & 0 & 0 & \multicolumn{2}{|c|}{day 10s digit} & \multicolumn{4}{|c|}{day 1s digit} \\ \hline
            \verb+0xD697+ & R/W & Days Alarm & 0 & 0 & \multicolumn{2}{|c|}{day 10s digit} & \multicolumn{4}{|c|}{day 1s digit} \\ \hline
            \verb+0xD698+ & R/W & Day of Week & 0 & 0 & 0 & 0 & 0 & \multicolumn{3}{|c|}{day of week digit} \\ \hline
            \verb+0xD699+ & R/W & Month & 0 & 0 & 0 & month 10s digit & \multicolumn{4}{|c|}{month 1s digit} \\ \hline
            \verb+0xD69A+ & R/W & Year & \multicolumn{4}{|c|}{year 10s digit} & \multicolumn{4}{|c|}{year 1s digit}  \\ \hline
            \verb+0xD69B+ & R/W & Rates & 0 & \multicolumn{3}{|c|}{WD} & \multicolumn{4}{|c|}{RS} \\ \hline
            \verb+0xD69C+ & R/W & Enables & 0 & 0 & 0 & 0 & AIE & PIE & PWRIE & ABE \\ \hline
            \verb+0xD69D+ & R/W & Flags & 0 & 0 & 0 & 0 & AF & PF & PWRF & BVF \\ \hline
            \verb+0xD69E+ & R/W & Control & 0 & 0 & 0 & 0 & UTI & STOP & 12/24 & DSE \\ \hline
            \verb+0xD69F+ & R/W & Century & \multicolumn{4}{|c|}{century 10s digit} & \multicolumn{4}{|c|}{century 1s digit}  \\ \hline
		\end{tabular}
	\end{center}
	\caption{Real Time Clock Registers}
	\label{tab:rtc_registers}
\end{table}

There are 16 registers for the RTC (see table~\ref{tab:rtc_registers}). There is a register each for century, year, month, day of the week ({\it i.e.} Sunday-Saturday), day, hour, minute, and second. Each one is expressed in binary-coded-decimal, meaning the lower four bits are the ones digit (0-9), and the upper bits are the 10s digit. In most cases, the upper digit is limited ({\it e.g.} seconds and minutes can only have 0-5 as the tens digit). For seconds, minutes, hours, and day there is a separate alarm register, which will be described later. Finally, there are the four registers for rates, enabled, flags, and control:

The Enables register has four separate enable bits:
\begin{description}
    \item[AIE] if set (1), the alarm interrupt will be enabled. The RTC will raise an interrupt when the current time matches the time specified in the alarm registers.

    \item[PIE] if set (1), the RTC will raise an interrupt periodically, where the period is specified by the RS field.

    \item[PWRIE] if set (1), the RTC will raise an interrupt on a power failure (not relevant to the \jr).

    \item[ABE] if set (1), the RTC will allow alarm interrupts when on battery backup (not relevant to the \jr).
\end{description}

The Flags register has four separate flags, which generally reflect why an interrupt was raised:
\begin{description}
    \item[AF] if set (1), the alarm was triggered

    \item[PF] if set (1), the periodic interrupt was triggered

    \item[PWRF] if set (1), the power failure interrupt was triggered

    \item[BVF] if set (1), the battery voltage is within safe range. If clear (0), the battery voltage is low, and the time may be invalid.
\end{description}

The Control register has four bits which change how the RTC operates:
\begin{description}
    \item[UTI] if set (1), the update of the externally facing registers by the internal timers is inhibited. In order to read or write those registers, the program must first set UTI and then clear it when done.

    \item[STOP] this bit allows for a battery saving feature. If it is clear (0) before the system is powered down, it will avoid draining the battery and may stop tracking the time. If it is set (1), it will keep using the battery as long as possible.

    \item[12/24] sets whether the RTC is using 12 or 24 hour accounting.

    \item[DSE] if set (1), daylight savings is in effect.
\end{description}

The Rates register controls the watchdog timer and the periodic interrupt. The watchdog timer is not really relevant to the \jr, but it monitors for activity and raises an interrupt if activity has not been seen within a certain amount of time (specified by the WD field). The periodic interrupt will be raised repeatedly, the period of which is set by the RS field (see table~\ref{tab:rtc_rs}).

\begin{table}[ht]
	\begin{center}
		\begin{tabular}{| c | c | c | c | c |} \hline
			RS3 & RS2 & RS1 & RS0 & Period \\ \hline\hline
			0 & 0 & 0 & 0 & None \\ \hline
            0 & 0 & 0 & 1 & 30.5175 \microsec \\ \hline
            0 & 0 & 1 & 0 & 61.035 \microsec \\ \hline
            0 & 0 & 1 & 1 & 122.070 \microsec \\ \hline
            0 & 1 & 0 & 0 & 244.141 \microsec \\ \hline
            0 & 1 & 0 & 1 & 488.281 \microsec \\ \hline
            0 & 1 & 1 & 0 & 976.5625 \microsec \\ \hline
            0 & 1 & 1 & 1 & 1.95315 ms \\ \hline
            1 & 0 & 0 & 0 & 3.90625 ms \\ \hline
            1 & 0 & 0 & 1 & 7.8125 ms \\ \hline
            1 & 0 & 1 & 0 & 15.625 ms \\ \hline
            1 & 0 & 1 & 1 & 31.25 ms \\ \hline
            1 & 1 & 0 & 0 & 62.5 ms \\ \hline
            1 & 1 & 0 & 1 & 125 ms \\ \hline
            1 & 1 & 1 & 0 & 250 ms \\ \hline
            1 & 1 & 1 & 1 & 500 ms \\ \hline
		\end{tabular}
	\end{center}
	\caption{RTC Periodic Interrupt Rates}
	\label{tab:rtc_rs}
\end{table}

\example{Display the Time}
In this example, we will read the time from the real time clock chip and print it out to the screen in {\it hh:mm:ss} format. The basic procedure is fairly simple: first the code disables the update of the transfer registers, then the code reads the hours and prints them, then the code reads the minutes and prints them, then the code fetches the seconds and prints them. Finally, the code re-enables the update of the transfer registers by dropping the UTI flag.

NOTE: This code resets the MMU I/O page to 0 before it tries to read from the clock chip. This is just to allow for the possibility of the kernel routines changing the I/O page without restoring it to 0.

\begin{verbatim}
ok_cint = $FF81							; OpenKernal call to initialize the screen
ok_cout = $FFD2							; OpenKernal call to print the character code in A

RTC_SECS = $D690						; RTC Seconds register
RTC_MINS = $D692						; RTC Minutes register
RTC_HOURS = $D694						; RTC Hours register

RTC_CTRL = $D96E						; RTC Control register
RTC_24HR = $02							; 12/24 hour flag (1 = 24 Hr, 0 = 12 Hr)
RTC_STOP = $04							; 0 = STOP when power off, 1 = run from battery when power off
RTC_UTI = $08							; Update Transfer Inhibit

start:      jsr ok_cint                 ; Initialize the text screen

            stz MMU_IO_CTRL             ; Make sure we're on I/O page 0

            lda RTC_CTRL                ; Stop the update of the RTC registers
            ora #RTC_UTI | RTC_24HR
            sta RTC_CTRL

            stz MMU_IO_CTRL				; Make sure we're on I/O page 0

            lda RTC_HOURS               ; Print the hours
            jsr putbcd

            lda #':'
            jsr ok_cout

            stz MMU_IO_CTRL				; Make sure we're on I/O page 0

            lda RTC_MINS                ; Print the minutes
            jsr putbcd

            lda #':'
            jsr ok_cout

            stz MMU_IO_CTRL				; Make sure we're on I/O page 0

            lda RTC_SECS                ; Print the seconds
            jsr putbcd

            stz MMU_IO_CTRL				; Make sure we're on I/O page 0

            lda RTC_CTRL                ; Reenable the update of the registers
            and #~RTC_UTI
            sta RTC_CTRL
\end{verbatim}

\nopagebreak[1] Since the time registers of the clock chip are encoded in binary-coded-decimal, printing is relatively straightforward, and is handled by a simple \verb+putbcd+ subroutine:

\begin{verbatim}
;
; Print a BCD number to the screen
;
putbcd:     pha                         ; Save the number
            and #$F0                    ; Isolate the upper digit
            lsr a
            lsr a
            lsr a
            lsr a

            clc                         ; Convert to ASCII
            adc #'0'
            jsr ok_cout                 ; And print

            pla                         ; Get the full number back
            and #$0F                    ; Isolate the lower digit

            clc                         ; Convert to ASCII
            adc #'0'
            jsr ok_cout                 ; And print

            rts
\end{verbatim}


\end{document}
