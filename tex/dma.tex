\chapter{Direct Memory Access}

The DMA engine can either write a specific byte to RAM or copy a set of bytes from one location in RAM to another. The DMA engine can also treat memory as being arranged either linearly (that is, as a certain number of consecutive locations) or rectangularly (the data is a rectangular area of an image).

\section{Linear Data}
Linear data (or ``1D'', if you prefer) is just a single block of sequential memory locations. When filling or copying data linearly, you need a destination address (and a source address if copying), and a count of bytes to copy. That is really all there is to it.

\section{Rectangular Data}
Rectangular data (or ``2D'') is a bit more complicated and is meant to be working with image data. With a bitmap, the pixel bytes are arranged in memory left to right and top to bottom. If the image starts at address $a$ and is $w$ pixels wide, then the pixel at $(x, y)$ can be found at location $a + y \times w + x$. Rectangular fills and copies are meant to work on data that is arranged in this fashion. In this case, you can use DMA to fill or copy a rectangular area within that image. As with linear fills and copies, you will need a destination address (and source address if doing a copy), but instead of a count of bytes you need the width and height of the rectangular areas affected. But you need one other thing, too. You need to tell the DMA the geometry of the over-all image... you need to tell it the width of the image containing the rectangular areas. This is called the ``stride'' and effectively tells the DMA how many pixels to skip between lines when it finishes one line of the rectangle before getting to the next line.

\begin{table}[h]
    \begin{center}
        \begin{tabular}{|c|c|c|c|c|c|c|c|c|c|} \hline
            Address & R/W & 7 & 6 & 5 & 4 & 3 & 2 & 1 & 0 \\\hline\hline
            \verb+0xDF00+ & R/W & START & \multicolumn{3}{|c|}{---} & INT\_EN & FILL & 2D & ENABLE \\ \hline
            \verb+0xDF01+ & W & FD7 & FD6 & FD5 & FD4 & FD3 & FD2 & FD1 & FD0 \\ \hline
            \verb+0xDF01+ & R & BUSY & \multicolumn{7}{|c|}{---}  \\ \hline\hline

            \verb+0xDF04+ & R/W & SA7 & SA6 & SA5 & SA4 & SA3 & SA2 & SA1 & SA0 \\ \hline
            \verb+0xDF05+ & R/W & SA15 & SA14 & SA13 & SA12 & SA11 & SA10 & SA9 & SA8 \\ \hline
            \verb+0xDF06+ & R/W & \multicolumn{6}{|c|}{---} & SA17 & SA16 \\ \hline\hline

            \verb+0xDF08+ & R/W & DA7 & DA6 & DA5 & DA4 & DA3 & DA2 & DA1 & DA0 \\ \hline
            \verb+0xDF09+ & R/W & DA15 & DA14 & DA13 & DA12 & DA11 & DA10 & DA9 & DA8 \\ \hline
            \verb+0xDF0A+ & R/W & \multicolumn{6}{|c|}{---} & DA17 & DA16 \\ \hline\hline

            \verb+0xDF0C+ & R/W & CNT7 & CNT6 & CNT5 & CNT4 & CNT3 & CNT2 & CNT1 & CNT0 \\ \hline
            \verb+0xDF0D+ & R/W & CNT15 & CNT14 & CNT13 & CNT12 & CNT11 & CNT10 & CNT9 & CNT8 \\ \hline
            \verb+0xDF0E+ & R/W & \multicolumn{6}{|c|}{---} & CNT17 & CNT16 \\ \hline\hline

            \verb+0xDF0C+ & R/W & R/W7 & R/W6 & R/W5 & R/W4 & R/W3 & R/W2 & R/W1 & R/W0 \\ \hline
            \verb+0xDF0D+ & R/W & R/W15 & R/W14 & R/W13 & R/W12 & R/W11 & R/W10 & R/W9 & R/W8 \\ \hline
            \verb+0xDF0E+ & R/W & H7 & H6 & H5 & H4 & H3 & H2 & H1 & H0 \\ \hline
            \verb+0xDF0F+ & R/W & H15 & H14 & H13 & H12 & H11 & H10 & H9 & H8 \\ \hline\hline

            \verb+0xDF10+ & R/W & SX7 & SX6 & SX5 & SX4 & SX3 & SX2 & SX1 & SX0 \\ \hline
            \verb+0xDF11+ & R/W & SX15 & SX14 & SX13 & SX12 & SX11 & SX10 & SX9 & SX8 \\ \hline

            \verb+0xDF12+ & R/W & SY7 & SY6 & SY5 & SY4 & SY3 & SY2 & SY1 & SY0 \\ \hline
            \verb+0xDF13+ & R/W & SY15 & SY14 & SY13 & SY12 & SY11 & SY10 & SY9 & SY8 \\ \hline
        \end{tabular}
    \end{center}
    \caption{DMA Registers}
    \label{tab:dma_reg}
\end{table}

\begin{description}
    \item[START] set to trigger the DMA

    \item[INT\_EN] enables triggering an interrupt when DMA is complete

    \item[FILL] when set, DMA will write a specific byte to memory. When clear, DMA will copy data from a source address to the destination address

    \item[2D] when set, DMA copies or fills a rectangular region of memory. When clear, DMA copies or fills a certain number of sequential bytes

    \item[ENABLE] set to enable DMA

    \item[FD] the byte to be written to memory when FILL is set

    \item[BUSY] status bit set when DMA is busy copying data

    \item[SA] the 18 bit source address (must be a location in the first 256KB of RAM). Only relevant when FILL is clear.

    \item[DA] the 18 bit destination address (must be a location in the first 256KB of RAM)

    \item[CNT] the number of bytes to copy (only available when 2D is clear)

    \item[W] the width of the rectangle of data to copy (only available when 2D is set)

    \item[H] the height of the rectangle of data to copy (only available when 2D is set)

    \item[SX] the width of the ``stride'' (only available when 2D is set)

    \item[SY] the height of the ``stride'' (only available when 2D is set)
\end{description}
