\chapter{Tiles}
The third graphics engine TinyVicky provides is the tile map system. The tile map system might seem a bit confusing at first, but really it is very similar to text mode, just made more flexible. In text mode, we have characters (256 of them). The shapes of the characters are defined in the font. What character is shown in a particular spot on the screen is set in the text matrix, which is a rectangular array of bytes in memory. In the same way, with the tile system we have tiles (256 of those, too). What those tiles look like are defined in a ``tile set.'' What tile is shown in a particular spot on the screen is set in the ``tile map.'' So there is an analogy:

\begin{eqnarray*}
    {\rm character} & \approx & {\rm tile} \\
    {\rm font} & \approx & {\rm tile\ set} \\
    {\rm text\ matrix} & \approx & {\rm tile\ map} \\
\end{eqnarray*}

There are several differences with tile maps, however:

\begin{itemize}
    \item A tile map may use tiles that are either $8 \times 8$ pixels or $16 \times 16$ pixels.

    \item A tile map can be scrolled smoothly horizontally or vertically.

    \item A tile may use 256 colors in its pixels as opposed to text mode's two-color characters. This means that a tile set uses one byte per pixel, with that byte's value being an index into a CLUT (as with bitmaps and sprites), where text mode fonts are one {\em bit} per pixel choosing between a foreground and background color.

    \item The tile map system allows for up to eight different tile sets to be used at the same time, where text mode has a single font.

    \item Up to three different tile maps can be displayed at one time, where text mode can only display one text matrix.

    \item A tile map can be placed on any one of three display layers, where text mode is always on top.
\end{itemize}


\section*{Tile Maps}

There are three tile maps supported by TinyVicky, each of which has 13 bytes worth of registers (see table:~\ref{tab:tilemap_reg}). Tile map 0 starts at 0xD200. Tile map 1 starts at 0xD20C. Tile map 2 starts at 0xD218.

\begin{table}[ht]
    \begin{center}
        \begin{tabular}{|c|c|c|c|c|c|c|c|c|c|c|} \hline
            Offset & R/W & 7 & 6 & 5 & 4 & 3 & 2 & 1 & 0 \\ \hline\hline
            0 & W & \multicolumn{3}{|c|}{---} & TILE\_SIZE & \multicolumn{3}{|c|}{---} & ENABLE \\ \hline
            1 & W & AD7 & AD6 & AD5 & AD4 & AD3 & AD2 & AD1 & AD0 \\ \hline
            2 & W & AD15 & AD14 & AD13 & AD12 & AD11 & AD10 & AD9 & AD8 \\ \hline
            3 & W & \multicolumn{6}{|c|}{---} & AD17 & AD16 \\ \hline
            4 & W & \multicolumn{8}{|c|}{MAP\_SIZE\_X} \\ \hline
            5 & W & \multicolumn{8}{|c|}{RESERVED} \\ \hline
            6 & W & \multicolumn{8}{|c|}{MAP\_SIZE\_Y} \\ \hline
            7 & W & \multicolumn{8}{|c|}{RESERVED} \\ \hline
            8 & W & X3 & X2 & X1 & X0 & SSX3 & SSX2 & SSX1 & SSX0 \\ \hline
            9 & W & DIR\_X & --- & X9 & X8 & X7 & X6 & X5 & X4 \\ \hline
            10 & W & Y3 & Y2 & Y1 & Y0 & SSY3 & SSY2 & SSY1 & SSY0 \\ \hline
            12 & W & DIR\_Y &\multicolumn{3}{|c|}{---} & Y7 & Y6 & Y5 & Y4 \\ \hline
        \end{tabular}
    \end{center}
    \caption{Tile Map Registers}
    \label{tab:tilemap_reg}
\end{table}

\begin{description}
    \item[TILE\_SIZE] if 0, tiles are 8 pixels wide by 8 tall. If 1, tiles are 16 pixels wide by 16 pixels tall

    \item[ENABLE] if set, the tile map will be displayed (if GRAPH and TILES are set in TinyVicky's Master Control Register)

    \item[AD] the address of the tile map data in RAM

    \item[MAP\_SIZE\_X] the width of the tile map in tiles ({\it i.e.} the number of columns)

    \item[MAP\_SIZE\_Y] the height of the tile map in tiles ({\it i.e.} the number of rows)

    \item[X] horizontal scroll in tile widths

    \item[SSX] horizontal scroll in pixels. How these bits are used varies with the size. If tiles are 16 pixels wide, then flags SSX[3..0] are used. If tiles are only 8 pixels wide, then only SSX[3..1] are used.

    \item[DIR\_X] the direction of the horizontal scroll.

    \item[Y] vertical scroll in tile heights

    \item[SSY] vertical scroll in pixels. How these bits are used varies with the size. If tiles are 16 pixels wide, then flags SSY[3..0] are used. If tiles are only 8 pixels wide, then only SSY[3..1] are used.

    \item[DIR\_Y] the direction of the vertical scroll.

\end{description}

One way tile maps get their flexibility is that, where text mode uses 8-bit bytes for the text matrix, a tile map is actually a rectangular collection of 16-bit integers in memory. A tile map entry is divided up into two pieces: the first byte is the number of the tile to display in that position (much like the character code in text mode), but the upper byte contains attribute bits (see table:~\ref{tab:tile_bits}), which have two fields:

\begin{description}
    \item[SET] is the number of the tile set to use for this tile's appearance

    \item[CLUT] is the number of the graphics CLUT to use in setting the colors
\end{description}

This attribute system makes tiles very powerful. Effectively, a single tile map can display 1,024 completely unique shapes at one time by using all eight tile sets. Also, since the CLUT is set for each tile in the attributes, the number of tiles needed can be reduced by clever use of recoloring.

\begin{table}[ht]
    \begin{center}
        \begin{tabular}{|c|c|c|c|c|c|c|c|c|c|c|c|c|c|c|c|} \hline
            15 & 14 & 13 & 12 & 11 & 10 & 9 & 8 & 7 & 6 & 5 & 4 & 3 & 2 & 1 & 0 \\ \hline\hline
            \multicolumn{3}{|c|}{---} & \multicolumn{2}{|c|}{CLUT} & \multicolumn{3}{|c|}{SET} & \multicolumn{8}{|c|}{TILE NUMBER} \\ \hline
        \end{tabular}
    \end{center}
    \caption{A Tile Map Entry}
    \label{tab:tile_bits}
\end{table}

\subsection*{Scrolling}

Tile maps can scroll across the screen both horizontally and vertically. The position of the tile map on the screen is controlled through the registers at offsets 8, 9, and 10. The horizontal position is controlled by DIR\_X, X, and SSX. The vertical position is controlled by DIR\_Y, Y, and SSY. The bits X and Y set the position in units of tiles. That is, the number in X[9..0] specifies how many complete tile columns the tile map is moved left or right. Likewise, Y[9..0] specifies how many tile rows the tile map is moved up or down. The SSX and SSY bits are used to specify how many rows of pixels within a tile the tile map is to move. SSX and SSY are therefore ``smooth scroll'' registers. They have a small trick to their use, however:

If the tile map uses tiles 16 pixels on a side, SSX[3..0] is used to specify the number of pixels to shift the tile map left or right: from 0 to 15. If, on the other hand, the tile map uses tiles 8 pixels on a side, only SSX[3..1] are used to specify the number of pixels to move: from 0 to 7. Note that SSX[0] is not used at all in this case. The SSY bits work in exactly the same way for smooth scrolling in the vertical direction.

Finally, DIR\_X and DIR\_Y are used to control the direction of the scrolling. If DIR\_X is 0, the tile map will move to the left. If DIR\_X is 1, the tile map will move to the right. If DIR\_Y is 0, the tile map will move to up. If DIR\_Y is 1, the tile map will move down. Note that the representation of the amount of the scrolling is separate from the direction (set X to 3 to scroll by 3 tiles, whether that is 3 to the left or 3 to the right). One way to look at the scroll registers is that they are one's complement numbers: a magnitude and a separate sign bit.

To make sure that scrolling will work properly, the tile map needs to be at least as big as the full screen (even if it is largely ``empty''), and there should be blank columns to the left and the right and blank rows above and below. That is, it is best to leave an empty margin all the way around your working tile map.

\section*{Tile Sets}

Essentially, a tile set is just a bitmap, but of a smaller size and arranged in a specific pattern. A tile set can be either a linear arrangement of tiles or a square arrangement of tiles. In the linear arrangement, the image is one tile wide by 256 tiles high. So for $8 \times 8$ tiles, the tile set is 8 pixels wide by 2,048 pixels high. For $16 \times 16$ tiles, the tile set is 16 pixels wide by 4,096 pixels high. The tiles are arranged vertically, so the first 8 or 16 (depending on tile size) rows are tile 0, the second set of rows are tile 1, and so on. For the square arrangement, the tile set is either 128 pixels wide by 128 pixels high (for $8 \times 8$ tiles), or it is 256 pixels wide by 256 pixels high (for $16 \times 16$ tiles). In both cases, the tiles are laid out left to right and top to bottom in a grid that is 16 tiles wide by 16 tiles high (see table~\ref{tab:tile_set_layout}).

As with bitmaps and sprites, the pixels of the tiles are each an individual byte. The contents of the byte (0 -- 255) serving as an index into a color lookup table. The pixels are also laid out in left-to-right and top-to-bottom order, just as with bitmaps and individual sprites.

\begin{table}[ht]
    \begin{center}
        \begin{tabular}{|c|c|c|c|c|c|c|c|c|c|c|c|c|c|c|c|c|} \hline
            & \multicolumn{16}{|c|}{128 or 256 pixels} \\ \hline
            \multirow{16}{*}{\rotatebox{90}{128 or 256 pixels}}
            & 0 & 1 & 2 & 3 & 4 & 5 & 6 & 7 & 8 & 9 & 10 & 11 & 12 & 13 & 14 & 15 \\ \cline{2-17}
            & 16 & 17 & 18 & 19 & 20 & 21 & 22 & 23 & 24 & 25 & 26 & 27 & 28 & 29 & 30 & 31 \\ \cline{2-17}
            & 32 & 33 & 34 & 35 & 36 & 37 & 38 & 39 & 40 & 41 & 42 & 43 & 44 & 45 & 46 & 46 \\ \cline{2-17}
            & 48 & 49 & 50 & 51 & 52 & 53 & 54 & 55 & 56 & 57 & 58 & 59 & 60 & 61 & 62 & 63 \\ \cline{2-17}
            & 64 & 65 & 66 & 67 & 68 & 69 & 60 & 71 & 72 & 73 & 74 & 75 & 76 & 77 & 78 & 79 \\ \cline{2-17}
            & 80 & 81 & 82 & 83 & 84 & 85 & 86 & 87 & 88 & 89 & 90 & 91 & 92 & 93 & 94 & 95 \\ \cline{2-17}
            & 96 & 97 & 98 & 99 & 100 & 101 & 102 & 103 & 104 & 105 & 106 & 107 & 108 & 109 & 110 & 111 \\ \cline{2-17}
            & 112 & 113 & 114 & 115 & 116 & 117 & 118 & 119 & 120 & 121 & 122 & 123 & 124 & 125 & 126 & 127 \\ \cline{2-17}
            & 128 & 129 & 130 & 131 & 132 & 133 & 134 & 135 & 136 & 137 & 138 & 139 & 140 & 141 & 142 & 143 \\ \cline{2-17}
            & 144 & 145 & 146 & 147 & 148 & 149 & 150 & 151 & 152 & 153 & 154 & 155 & 156 & 157 & 158 & 159 \\ \cline{2-17}
            & 160 & 161 & 162 & 163 & 164 & 165 & 166 & 167 & 168 & 169 & 170 & 171 & 172 & 173 & 174 & 175 \\ \cline{2-17}
            & 176 & 177 & 178 & 179 & 180 & 181 & 182 & 183 & 184 & 185 & 186 & 187 & 188 & 189 & 190 & 191 \\ \cline{2-17}
            & 192 & 193 & 194 & 195 & 196 & 197 & 198 & 199 & 200 & 201 & 202 & 203 & 204 & 205 & 206 & 207 \\ \cline{2-17}
            & 208 & 209 & 210 & 211 & 212 & 213 & 214 & 215 & 216 & 217 & 218 & 219 & 220 & 221 & 222 & 223 \\ \cline{2-17}
            & 224 & 225 & 226 & 227 & 228 & 229 & 230 & 231 & 232 & 233 & 234 & 235 & 236 & 237 & 238 & 239 \\ \cline{2-17}
            & 240 & 241 & 242 & 243 & 244 & 245 & 246 & 247 & 248 & 249 & 250 & 251 & 252 & 253 & 254 & 255 \\ \hline
        \end{tabular}
    \end{center}
    \caption{Arrangement of Tiles in a Tile Set Image}
    \label{tab:tile_set_layout}
\end{table}

TinyVicky supports eight separate tile sets. Each one has a single three byte address register, which provides the address to the tile set pixel data, and a configuration register (see table:~\ref{tab:tile_set_addr}). To use them, a program simply stores the address of the pixel data to use into the appropriate address register. The configuration register contains a single SQUARE flag, which indicates the layout of the tile set image. If SQUARE is set (1), the tile set image is square ($128 \times 128$ pixels for $8 \times 8$ tiles or $256 \times 256$ pixels for $16 \times 16$ tiles). If SQUARE is clear (0), the tile set image is vertical ($8 \times 2,048$ pixels for $8 \times 8$ tiles, or $16 \times 4,096$ pixels for $16 \times 16$ tiles).

\begin{table}[ht]
    \begin{center}
        \begin{tabular}{|c|c|c|c|c|c|c|c|c|c|c|} \hline
            Address & R/W & Tile Set & 7 & 6 & 5 & 4 & 3 & 2 & 1 & 0 \\ \hline\hline
            \verb+0xD280+ & W & \multirow{4}{*}{0} & AD7 & AD6 & AD5 & AD4 & AD3 & AD2 & AD1 & AD0 \\ \cline{1-2}\cline{4-11}
            \verb+0xD281+ & W &                    & AD15 & AD14 & AD13 & AD12 & AD11 & AD10 & AD9 & AD8 \\ \cline{1-2}\cline{4-11}
            \verb+0xD282+ & W &                    & \multicolumn{6}{|c|}{---} & AD17 & AD16 \\ \cline{1-2}\cline{4-11}
            \verb+0xD283+ & W &                    & \multicolumn{4}{|c|}{---} & SQUARE & \multicolumn{3}{|c|}{---} \\ \hline\hline

            \verb+0xD284+ & W & \multirow{4}{*}{1} & AD7 & AD6 & AD5 & AD4 & AD3 & AD2 & AD1 & AD0 \\ \cline{1-2}\cline{4-11}
            \verb+0xD285+ & W &                    & AD15 & AD14 & AD13 & AD12 & AD11 & AD10 & AD9 & AD8 \\ \cline{1-2}\cline{4-11}
            \verb+0xD286+ & W &                    & \multicolumn{6}{|c|}{---} & AD17 & AD16 \\ \cline{1-2}\cline{4-11}
            \verb+0xD287+ & W &                    & \multicolumn{4}{|c|}{---} & SQUARE & \multicolumn{3}{|c|}{---} \\ \hline\hline

            \verb+0xD288+ & W & \multirow{4}{*}{2} & AD7 & AD6 & AD5 & AD4 & AD3 & AD2 & AD1 & AD0 \\ \cline{1-2}\cline{4-11}
            \verb+0xD289+ & W &                    & AD15 & AD14 & AD13 & AD12 & AD11 & AD10 & AD9 & AD8 \\ \cline{1-2}\cline{4-11}
            \verb+0xD28A+ & W &                    & \multicolumn{6}{|c|}{---} & AD17 & AD16 \\ \cline{1-2}\cline{4-11}
            \verb+0xD28B+ & W &                    & \multicolumn{4}{|c|}{---} & SQUARE & \multicolumn{3}{|c|}{---} \\ \hline\hline

            \verb+0xD28C+ & W & \multirow{4}{*}{3} & AD7 & AD6 & AD5 & AD4 & AD3 & AD2 & AD1 & AD0 \\ \cline{1-2}\cline{4-11}
            \verb+0xD28D+ & W &                    & AD15 & AD14 & AD13 & AD12 & AD11 & AD10 & AD9 & AD8 \\ \cline{1-2}\cline{4-11}
            \verb+0xD28E+ & W &                    & \multicolumn{6}{|c|}{---} & AD17 & AD16 \\ \cline{1-2}\cline{4-11}
            \verb+0xD28F+ & W &                    & \multicolumn{4}{|c|}{---} & SQUARE & \multicolumn{3}{|c|}{---} \\ \hline\hline

            \verb+0xD290+ & W & \multirow{4}{*}{4} & AD7 & AD6 & AD5 & AD4 & AD3 & AD2 & AD1 & AD0 \\ \cline{1-2}\cline{4-11}
            \verb+0xD291+ & W &                    & AD15 & AD14 & AD13 & AD12 & AD11 & AD10 & AD9 & AD8 \\ \cline{1-2}\cline{4-11}
            \verb+0xD292+ & W &                    & \multicolumn{6}{|c|}{---} & AD17 & AD16 \\ \cline{1-2}\cline{4-11}
            \verb+0xD293+ & W &                    & \multicolumn{4}{|c|}{---} & SQUARE & \multicolumn{3}{|c|}{---} \\ \hline\hline

            \verb+0xD294+ & W & \multirow{4}{*}{5} & AD7 & AD6 & AD5 & AD4 & AD3 & AD2 & AD1 & AD0 \\ \cline{1-2}\cline{4-11}
            \verb+0xD295+ & W &                    & AD15 & AD14 & AD13 & AD12 & AD11 & AD10 & AD9 & AD8 \\ \cline{1-2}\cline{4-11}
            \verb+0xD296+ & W &                    & \multicolumn{6}{|c|}{---} & AD17 & AD16 \\ \cline{1-2}\cline{4-11}
            \verb+0xD297+ & W &                    & \multicolumn{4}{|c|}{---} & SQUARE & \multicolumn{3}{|c|}{---} \\ \hline\hline

            \verb+0xD298+ & W & \multirow{4}{*}{6} & AD7 & AD6 & AD5 & AD4 & AD3 & AD2 & AD1 & AD0 \\ \cline{1-2}\cline{4-11}
            \verb+0xD299+ & W &                    & AD15 & AD14 & AD13 & AD12 & AD11 & AD10 & AD9 & AD8 \\ \cline{1-2}\cline{4-11}
            \verb+0xD29A+ & W &                    & \multicolumn{6}{|c|}{---} & AD17 & AD16 \\ \cline{1-2}\cline{4-11}
            \verb+0xD29B+ & W &                    & \multicolumn{4}{|c|}{---} & SQUARE & \multicolumn{3}{|c|}{---} \\ \hline\hline

            \verb+0xD29C+ & W & \multirow{4}{*}{7} & AD7 & AD6 & AD5 & AD4 & AD3 & AD2 & AD1 & AD0 \\ \cline{1-2}\cline{4-11}
            \verb+0xD29D+ & W &                    & AD15 & AD14 & AD13 & AD12 & AD11 & AD10 & AD9 & AD8 \\ \cline{1-2}\cline{4-11}
            \verb+0xD29E+ & W &                    & \multicolumn{6}{|c|}{---} & AD17 & AD16 \\ \cline{1-2}\cline{4-11}
            \verb+0xD29F+ & W &                    & \multicolumn{4}{|c|}{---} & SQUARE & \multicolumn{3}{|c|}{---} \\ \hline
        \end{tabular}
    \end{center}
    \caption{Tile Set Registers}
    \label{tab:tile_set_addr}
\end{table}

\example{A Simple Tile Map}

\begin{verbatim}
            ;
            ; Set up TinyVicky to display tiles
            ;
            lda #$14                    ; Graphics and Tile engines enabled
            sta VKY_MSTR_CTRL_0
            stz VKY_MSTR_CTRL_1         ; 320x240 @ 60Hz

            lda #$40                    ; Layer 0 = Bitmap 0, Layer 1 = Tile map 0
            sta VKY_LAYER_CTRL_0
            lda #$15                    ; Layer 2 = Tile Map 1
            sta VKY_LAYER_CTRL_1

            stz VKY_BRDR_CTRL           ; No border

            lda #$19                    ; Background: midnight blue
            sta VKY_BKG_COL_R
            lda #$19
            sta VKY_BKG_COL_G
            lda #$70
            sta VKY_BKG_COL_B
\end{verbatim}

To define the tile set, all we really need to do is to set the address register for the tile set to point to the actual pixel data. In this particular case, the code is just going to use tile set 0.

\begin{verbatim}
            ;
            ; Set tile set #0 to our image
            ;

            lda #<tiles_img_start
            sta VKY_TS0_ADDR_L
            lda #>tiles_img_start
            sta VKY_TS0_ADDR_M
            lda #`tiles_img_start
            sta VKY_TS0_ADDR_H
\end{verbatim}

Finally, the code sets up the tile map itself, setting the size of the tiles, the size of the tile map, setting the position of the screen in the tile map, and pointing to the tile map data.

\begin{verbatim}
            ;
            ; Set tile map #0
            ;

            lda #$01                    ; 16x16 tiles, enable
            sta VKY_TM0_CTRL

            lda #22                     ; Our tile map is 20x15
            sta VKY_TM0_SIZE_X
            lda #16
            sta VKY_TM0_SIZE_Y

            lda #<tile_map              ; Point to the tile map
            sta VKY_TM0_ADDR_L
            lda #>tile_map
            sta VKY_TM0_ADDR_M
            lda #`tile_map
            sta VKY_TM0_ADDR_H

            lda #$0F                    ; Set scrolling (15, 0)
            sta VKY_TM0_POS_X_L
            lda #$00
            sta VKY_TM0_POS_X_H

            stz VKY_TM0_POS_Y_L
            stz VKY_TM0_POS_Y_H
\end{verbatim}

The tile map itself. In this case, we just define it in-line. The data is formatted to match the dimensions of the tile map for ease of reading. Note that the left-most and right-most columns are essentially blank, providing some buffer space to allow for scrolling. Similarly, there is a spare row on the bottom. This data is formatted as single hexadecimal digits, to make it easier to format this data on the page, but the data is actually stored as 16-bit values. This is taking advantage of the fact that the code is using CLUT 0 and LAYER 0 for the tiles and that there are no more than 16 tiles in the tile set.

\begin{verbatim}
tile_map:
.word $4,$1,$0,$1,$0,$0,$0,$0,$0,$0,$0,$0,$0,$0,$0,$0,$0,$0,$4,$0,$4,$0
.word $0,$0,$1,$0,$0,$0,$0,$0,$0,$0,$0,$0,$0,$0,$0,$0,$0,$0,$0,$4,$0,$0
.word $0,$1,$0,$1,$0,$0,$6,$7,$7,$7,$7,$7,$7,$7,$7,$8,$0,$0,$4,$0,$4,$0
.word $0,$0,$0,$0,$0,$0,$9,$1,$2,$3,$4,$5,$0,$0,$0,$A,$0,$0,$0,$0,$0,$0
.word $0,$0,$0,$0,$0,$0,$9,$2,$1,$2,$3,$4,$5,$0,$0,$A,$0,$0,$0,$0,$0,$0
.word $0,$0,$0,$0,$0,$0,$9,$3,$2,$1,$2,$3,$4,$5,$0,$A,$0,$0,$0,$0,$0,$0
.word $0,$0,$0,$0,$0,$0,$9,$4,$3,$2,$1,$2,$3,$4,$5,$A,$0,$0,$0,$0,$0,$0
.word $0,$0,$0,$0,$0,$0,$9,$5,$4,$3,$2,$1,$2,$3,$4,$A,$0,$0,$0,$0,$0,$0
.word $0,$0,$0,$0,$0,$0,$9,$0,$5,$4,$3,$2,$1,$2,$3,$A,$0,$0,$0,$0,$0,$0
.word $0,$0,$0,$0,$0,$0,$9,$0,$0,$5,$4,$3,$2,$1,$2,$A,$0,$0,$0,$0,$0,$0
.word $0,$0,$0,$0,$0,$0,$9,$0,$0,$0,$5,$4,$3,$2,$1,$A,$0,$0,$0,$0,$0,$0
.word $0,$0,$0,$0,$0,$0,$B,$C,$C,$C,$C,$C,$C,$C,$C,$D,$0,$0,$0,$0,$0,$0
.word $0,$3,$0,$3,$0,$0,$0,$0,$0,$0,$0,$0,$0,$0,$0,$0,$0,$0,$2,$0,$2,$0
.word $0,$0,$3,$0,$0,$0,$0,$0,$0,$0,$0,$0,$0,$0,$0,$0,$0,$0,$0,$2,$0,$0
.word $0,$3,$0,$3,$0,$0,$0,$0,$0,$0,$0,$0,$0,$0,$0,$0,$0,$0,$2,$0,$2,$0
.word $0,$0,$0,$0,$0,$0,$0,$0,$0,$0,$0,$0,$0,$0,$0,$0,$0,$0,$2,$0,$0,$4
\end{verbatim}
