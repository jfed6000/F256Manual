\chapter{Tracking Time}

\section{Interval Timers}

\section{Real Time Clock}

For programs needing to keep track of time, \jr\ provides a real time clock chip (RTC), the bq4802. This chip, keeps track of the year (including century), month, day, hour (in 12 or 24 hour mode), minute, and second. The coin cell battery on the \jr\ motherboard is to provide power to the RTC so it can continue tracking time even when the \jr\ is turned off or unplugged. Additionally, the RTC can send interrupts to the CPU, either periodically or at a specific time.

The RTC is relatively straightforward to use, but one potentially tricky thing to keep in mind is that there is a specific procedure to follow when reading or writing the date-time. As well as the registers the CPU can access, the RTC has internal registers which are constantly updating as time progresses. Normally, the internal registers update their external counterparts, but this should not be allowed to happen while the CPU is getting or setting the externally facing registers. So, to access the external registers, the program must first disable the automatic updates to the external registers. Then it can read or write the external registers. Then it can re-enable the automatic updates. If the program has changed the registers, when updates are re-enabled the data in the external registers will be sent to the internal registers in one action. This keeps the time information consistent.

\begin{table}[h]
	\begin{center}
		\begin{tabular}{| c | c | c || c | c | c | c | c | c | c | c |} \hline
			Address & R/W & Name & 7 & 6 & 5 & 4 & 3 & 2 & 1 & 0 \\ \hline\hline
			\verb+0xD690+ & R/W & Seconds & 0 & \multicolumn{3}{|c|}{second 10s digit} & \multicolumn{4}{|c|}{second 1s digit} \\ \hline
            \verb+0xD691+ & R/W & Seconds Alarm & 0 & \multicolumn{3}{|c|}{second 10s digit} & \multicolumn{4}{|c|}{second 1s digit} \\ \hline
            \verb+0xD692+ & R/W & Minutes & 0 & \multicolumn{3}{|c|}{minute 10s digit} & \multicolumn{4}{|c|}{minute 1s digit} \\ \hline
            \verb+0xD693+ & R/W & Minutes Alarm & 0 & \multicolumn{3}{|c|}{minute 10s digit} & \multicolumn{4}{|c|}{minute 1s digit} \\ \hline
            \verb+0xD694+ & R/W & Hours & AM/PM & 0 & \multicolumn{2}{|c|}{hour 10s digit} & \multicolumn{4}{|c|}{hour 1s digit} \\ \hline
            \verb+0xD695+ & R/W & Hours Alarm & AM/PM & 0 & \multicolumn{2}{|c|}{hour 10s digit} & \multicolumn{4}{|c|}{hour 1s digit} \\ \hline
            \verb+0xD696+ & R/W & Days & 0 & 0 & \multicolumn{2}{|c|}{day 10s digit} & \multicolumn{4}{|c|}{day 1s digit} \\ \hline
            \verb+0xD697+ & R/W & Days Alarm & 0 & 0 & \multicolumn{2}{|c|}{day 10s digit} & \multicolumn{4}{|c|}{day 1s digit} \\ \hline
            \verb+0xD698+ & R/W & Day of Week & 0 & 0 & 0 & 0 & 0 & \multicolumn{3}{|c|}{day of week digit} \\ \hline
            \verb+0xD699+ & R/W & Month & 0 & 0 & 0 & month 10s digit & \multicolumn{4}{|c|}{month 1s digit} \\ \hline
            \verb+0xD69A+ & R/W & Year & \multicolumn{4}{|c|}{year 10s digit} & \multicolumn{4}{|c|}{year 1s digit}  \\ \hline
            \verb+0xD69B+ & R/W & Rates & 0 & \multicolumn{3}{|c|}{WD} & \multicolumn{4}{|c|}{RS} \\ \hline
            \verb+0xD69C+ & R/W & Enables & 0 & 0 & 0 & 0 & AIE & PIE & PWRIE & ABE \\ \hline
            \verb+0xD69D+ & R/W & Flags & 0 & 0 & 0 & 0 & AF & PF & PWRF & BVF \\ \hline
            \verb+0xD69E+ & R/W & Control & 0 & 0 & 0 & 0 & UTI & STOP & 12/24 & DSE \\ \hline
            \verb+0xD69F+ & R/W & Century & \multicolumn{4}{|c|}{century 10s digit} & \multicolumn{4}{|c|}{century 1s digit}  \\ \hline
		\end{tabular}
	\end{center}
	\caption{MMU Registers}
	\label{tab:rtc_registers}
\end{table}

There are 16 registers for the RTC (see table~\ref{tab:rtc_registers}). There is a register each for century, year, month, day of the week ({\it i.e.} Sunday - Saturday), day, hour, minute, and second. Each one is expressed in binary-coded-decimal, meaning the lower four bits are the ones digit (0 - 9), and the upper bits are the 10s digit. In most cases, the upper digit is limited ({\it e.g.} seconds and minutes can only have 0 - 6 as the tens digit). For seconds, minutes, hours, and day there is a separate alarm register, which will be described later. Finally, there are the four registers for rates, enabled, flags, and control:

The Enables register has four separate enable bits:
\begin{description}
    \item[AIE] if set (1), the alarm interrupt will be enabled. The RTC will raise an interrupt when the current time matches the time specified in the alarm registers.

    \item[PIE] if set (1), the RTC will raise an interrupt periodically, where the period is specified by the RS field.

    \item[PWRIE] if set (1), the RTC will raise an interrupt on a power failure (not relevant to the \jr).

    \item[ABE] if set (1), the RTC will allow alarm interrupts when on battery backup (not relevant to the \jr).
\end{description}

The Flags register has four separate flags, which generally reflect why an interrupt was raised:
\begin{description}
    \item[AF] if set (1), the alarm was triggered

    \item[PF] if set (1), the periodic interrupt was triggered

    \item[PWRF] if set (1), the power failure interrupt was triggered

    \item[BVF] if set (1), the battery voltage is within safe range. If clear (0), the battery voltage is low, and the time may be invalid.
\end{description}

The Control register has four bits which change how the RTC operates:
\begin{description}
    \item[UTI] if set (1), the update of the externally facing registers by the internal timers is inhibited. In order to read or write those registers, the program must first set UTI and then clear it when done.

    \item[STOP] this bit allows for a battery saving feature. If it is clear (0) before the system is powered down, it will avoid draining the battery and may stop tracking the time. If it is set (1), it will keep using the battery as long as possible.

    \item[12/24] sets whether the RTC is using 12 or 24 hour accounting.

    \item[DSE] if set (1), daylight savings is in effect.
\end{description}

The Rates register controls the watchdog timer and the periodic interrupt. The watchdog timer is not really relevant to the \jr, but it monitors for activity and raises an interrupt if activity has not been seen within a certain amount of time (specified by the WD field). The periodic interrupt will be raised repeatedly, the period of which is set by the RS field (see table~\ref{tab:rtc_rs}).

\begin{table}[h]
	\begin{center}
		\begin{tabular}{| c | c | c | c | c |} \hline
			RS3 & RS2 & RS1 & RS0 & Period \\ \hline\hline
			0 & 0 & 0 & 0 & None \\ \hline
            0 & 0 & 0 & 1 & 30.5175 \microsec \\ \hline
            0 & 0 & 1 & 0 & 61.035 \microsec \\ \hline
            0 & 0 & 1 & 1 & 122.070 \microsec \\ \hline
            0 & 1 & 0 & 0 & 244.141 \microsec \\ \hline
            0 & 1 & 0 & 1 & 488.281 \microsec \\ \hline
            0 & 1 & 1 & 0 & 976.5625 \microsec \\ \hline
            0 & 1 & 1 & 1 & 1.95315 ms \\ \hline
            1 & 0 & 0 & 0 & 3.90625 ms \\ \hline
            1 & 0 & 0 & 1 & 7.8125 ms \\ \hline
            1 & 0 & 1 & 0 & 15.625 ms \\ \hline
            1 & 0 & 1 & 1 & 31.25 ms \\ \hline
            1 & 1 & 0 & 0 & 62.5 ms \\ \hline
            1 & 1 & 0 & 1 & 125 ms \\ \hline
            1 & 1 & 1 & 0 & 250 ms \\ \hline
            1 & 1 & 1 & 1 & 500 ms \\ \hline
		\end{tabular}
	\end{center}
	\caption{RTC Periodic Interrupt Rates}
	\label{tab:rtc_rs}
\end{table}

\example{Display the Time}
In this example, we will read the time from the real time clock chip and print it out to the screen in {\it hh:mm:ss} format. To make things a little different, the code will also use the OpenKernal calls to initialize the text screen and print text. OpenKernel is a clean-room implementation of the Commodore Kernel calls. The basic procedure is fairly simple: first the code disables the update of the transfer registers, then the code reads the hours and prints them, then the code reads the minutes and prints them, then the code fetches the seconds and prints them. Finally, the code re-enables the update of the transfer registers by dropping the UTI flag.

NOTE: This code resets the MMU I/O page to 0 before it tries to read from the clock chip. This is just to allow for the possibility of the kernel routines changing the I/O page without restoring it to 0.

\begin{verbatim}
ok_cint = $FF81							; OpenKernal call to initialize the screen
ok_cout = $FFD2							; OpenKernal call to print the character code in A

RTC_SECS = $D690						; RTC Seconds register
RTC_MINS = $D692						; RTC Minutes register
RTC_HOURS = $D694						; RTC Hours register

RTC_CTRL = $D96E						; RTC Control register
RTC_24HR = $02							; 12/24 hour flag (1 = 24 Hr, 0 = 12 Hr)
RTC_STOP = $04							; 0 = STOP when power off, 1 = run from battery when power off
RTC_UTI = $08							; Update Transfer Inhibit

start:      jsr ok_cint                 ; Initialize the text screen

            stz MMU_IO_CTRL             ; Make sure we're on I/O page 0

            lda RTC_CTRL                ; Stop the update of the RTC registers
            ora #RTC_UTI | RTC_24HR
            sta RTC_CTRL

            stz MMU_IO_CTRL				; Make sure we're on I/O page 0

            lda RTC_HOURS               ; Print the hours
            jsr putbcd

            lda #':'
            jsr ok_cout

            stz MMU_IO_CTRL				; Make sure we're on I/O page 0

            lda RTC_MINS                ; Print the minutes
            jsr putbcd

            lda #':'
            jsr ok_cout

            stz MMU_IO_CTRL				; Make sure we're on I/O page 0

            lda RTC_SECS                ; Print the seconds
            jsr putbcd

            stz MMU_IO_CTRL				; Make sure we're on I/O page 0

            lda RTC_CTRL                ; Reenable the update of the registers
            and #~RTC_UTI
            sta RTC_CTRL
\end{verbatim}

Since the time registers of the clock chip are encoded in binary-coded-decimal, printing is relatively straightforward, and is handled by a simple \verb+putbcd+ subroutine:

\begin{verbatim}
;
; Print a BCD number to the screen
;
putbcd:     pha                         ; Save the number
            and #$F0                    ; Isolate the upper digit
            lsr a
            lsr a
            lsr a
            lsr a

            clc                         ; Convert to ASCII
            adc #'0'
            jsr ok_cout                 ; And print

            pla                         ; Get the full number back
            and #$0F                    ; Isolate the lower digit

            clc                         ; Convert to ASCII
            adc #'0'
            jsr ok_cout                 ; And print

            rts
\end{verbatim}
