\chapter{Interrupt Controller}

The 65C02 has two interrupts: non-maskable interrupts (NMI) for high priority events, and the regular interrupt request line (IRQ) for normal priority events. Currently, the C256 series of computers do not use NMI for any purpose, so the only interrupt is the IRQ line. There are many devices on the \jr\ which can trigger interrupts, so to save the interrupt handler the chore of querying each device in turn, the \jr\ provides an interrupt controller module.

The individual devices route their interrupt request signals to the interrupt controller. When an interrupt comes in, the controller knows which device it is and decides whether to forward the interrupt to the CPU. The interrupt handler can then query the interrupt handler to see which device or devices have interrupts pending and can then acknowledge them once they have been properly handled.

The interrupt controller provides two sets of registers to provide its functionality: interrupt masks, and interrupt pending. The mask registers provide a mask flag for every possible interrupt. If the flag is true (1), the interrupt from that device is masked, and an interrupt coming in from it will not trigger an IRQ on the processor. The pending register provides a pending flag for every possible interrupt. If the flag is true (1) on a read, the device has requested an interrupt, if false (0) there is no interrupt pending from the device. Setting a flag in a pending register to 1 will acknowledge the interrupt and cause the controller to drop the pending flag. Writing a 0 to a flag will have no effect.

The interrupt controller registers are divided on the \jr\ into two groups: 0 and 1. Group 0 represents seven of the interrupts: two video interrupts, two PS/2 controller interrupts, two timer interrupts, and the DMA interrupt. Group 1 represents the other interrupts: UART, real time clock, VIA, and the SD card controller.

NOTE: Some devices on the \jr\ have their own interrupt enable flags (separate from the mask flags). For example, the 65C22 VIA has an interrupt enable bit in one of its registers and will not send an interrupt to the \jr's interrupt controller if that bit is not enabled. For such devices, the interrupt enable flag on the device must be set and the corresponding mask bit in the interrupt controller must be clear in order for interrupts to be sent to the CPU. Other devices, like VICKY, do not have a separate enable flag. In their case, only their corresponding mask bits must be cleared to enable their interrupts.

\begin{table}[ht]
	\begin{center}
		\begin{tabular}{| c | c | c | c | l |} \hline
            Mask & Pending & Bit & Name & Purpose \\ \hline\hline
            \multirow{8}{*}{{\tt 0xD66C}} & \multirow{8}{*}{{\tt 0xD660}} & \verb+0x01+ & INT\_VKY\_SOF & TinyVicky Start Of Frame interrupt. \\ \cline{3-5}
            & & \verb+0x02+ & INT\_VKY\_SOL & TinyVicky Start Of Line interrupt \\ \cline{3-5}
            & & \verb+0x04+ & INT\_PS2\_KBD & PS/2 keyboard event \\ \cline{3-5}
            & & \verb+0x08+ & INT\_PS2\_MOUSE & PS/2 mouse event \\ \cline{3-5}
            & & \verb+0x10+ & INT\_TIMER\_0 & TIMER0 has reached its target value \\ \cline{3-5}
            & & \verb+0x20+ & INT\_TIMER\_1 & TIMER1 has reached its target value \\ \cline{3-5}
            & & \verb+0x40+ & RESERVED & \\ \cline{3-5}
            & & \verb+0x80+ & RESERVED & \\ \hline
        \end{tabular}
    \end{center}
	\caption{Interrupts Group 0}
	\label{tab:interrupts}
\end{table}

\begin{table}[ht]
	\begin{center}
		\begin{tabular}{| c | c | c | c | l |} \hline
            Mask & Pending & Bit & Name & Purpose \\ \hline\hline
            \multirow{8}{*}{{\tt 0xD66D}} & \multirow{8}{*}{{\tt 0xD661}} & \verb+0x01+ & INT\_UART & The UART is ready to receive or send data \\ \cline{3-5}
            & & \verb+0x02+ & RESERVED & \\ \cline{3-5}
            & & \verb+0x04+ & RESERVED & \\ \cline{3-5}
            & & \verb+0x08+ & RESERVED & \\ \cline{3-5}
            & & \verb+0x10+ & INT\_RTC & Event from the real time clock chip \\ \cline{3-5}
            & & \verb+0x20+ & INT\_VIA & Event from the 65C22 VIA chip \\ \cline{3-5}
            & & \verb+0x40+ & INT\_IEC & Event from the IEC serial bus \\ \cline{3-5}
            & & \verb+0x80+ & INT\_SDC\_INS & User has inserted an SD card \\ \hline
        \end{tabular}
    \end{center}
	\caption{Interrupts Group 1}
	\label{tab:interrupts_1}
\end{table}

\begin{table}[ht]
	\begin{center}
		\begin{tabular}{| c | c | c | c | l |} \hline
            Mask & Pending & Bit & Name & Purpose \\ \hline\hline
            \multirow{8}{*}{{\tt 0xD66E}} & \multirow{8}{*}{{\tt 0xD662}} & \verb+0x01+ & IEC\_DATA\_i & IEC Data In \\ \cline{3-5}
            & & \verb+0x02+ & IEC\_CLK\_i & IEC Clock In  \\ \cline{3-5}
            & & \verb+0x04+ & IEC\_ATN\_i & IEC ATN In  \\ \cline{3-5}
            & & \verb+0x08+ & IEC\_SREQ\_i & IEC REQ In  \\ \cline{3-5}
            & & \verb+0x10+ & RESERVED & \\ \cline{3-5}
            & & \verb+0x20+ & RESERVED & \\ \cline{3-5}
            & & \verb+0x40+ & RESERVED & \\ \cline{3-5}
            & & \verb+0x80+ & RESERVED & \\ \hline
        \end{tabular}
    \end{center}
	\caption{Interrupts Group 2}
	\label{tab:interrupts_2}
\end{table}

The Start Of Frame (SOF) and Start of Line (SOL) interrupts could use some further explanation. The SOF interrupt is raised at the beginning of the vertical blanking period, when the raster has reached the bottom of the screen and starts to return to the top. This interrupt is raised either 60 times a second or 70 times a second, depending on the value of CLK\_70 (see table~\ref{tab:vky_master_ctrl_reg}), which sets the base resolution of the screen. The SOF interrupt is good for timing updates to graphics (like placement of sprites) to avoid screen tearing. It can also be used for rough timing of events, provided the code takes into account the fact that the timing changes with screen resolution. The Start of Line interrupt is raised when the raster line has reached a target line (see table~\ref{tab:lint_reg}). When the interrupt is raised, the raster is in the process of drawing the screen and has reached the desired target line.

As an example of working with the interrupt controller, let's try using the SOF interrupt to alter the character in the upper left corner.

To start, we will need to install our interrupt handler to respond to IRQs. For this example, we're going to completely take over interrupt processing, so we'll do some things we wouldn't ordinarily do. Also, since an interrupt could come in while we're setting things, up, we need to be careful about how we do things.

\begin{enumerate}
    \item First, we want to disable IRQs at the CPU level.
    \item Then we set the interrupt vector.
    \item Next, we want to mask off all but the SOF interrupt, since that is the only one we will process (in real programs, we will either need to handle several interrupts or play nicely with the operating system).
    \item Now, there might be interrupts that came in earlier, so we will just clear all the pending interrupt flags to ensure the program starts cleanly.
    \item Finally, we enable CPU interrupt handling again and loop forever... processing the SOF interrupt when it comes in.
\end{enumerate}

\begin{verbatim}
VIRQ = $FFFE

INT_PEND_0 = $D660  ; Pending register for interrupts 0 - 7
INT_PEND_1 = $D661  ; Pending register for interrupts 8 - 15
INT_MASK_0 = $D66C  ; Mask register for interrupts 0 - 7
INT_MASK_1 = $D66D  ; Mask register for interrupts 8 - 15

start:      ; Disable IRQ handling
            sei

            ; Load my IRQ handler into the IRQ vector
            ; NOTE: this code just takes over IRQs completely. It could save
            ;       the pointer to the old handler and chain to it when it had
            ;       handled its interrupt. But what is proper really depends on
            ;       what the program is trying to do.
            lda #<my_handler
            sta VIRQ
            lda #>my_handler
            sta VIRQ+1

            ; Mask off all but the SOF interrupt
            lda #$ff
            sta INT_MASK_1
            and #~INT00_VKY_SOF
            sta INT_MASK_0

            ; Clear all pending interrupts
            lda #$ff
            sta INT_PEND_0
            sta INT_PEND_1

            ; Put a character in the upper right of the screen
            lda #SYS_CTRL_TEXT_PG
            sta SYS_CTRL_1

            lda #'@'
            sta $c000

            ; Set the color of the character
            lda #SYS_CTRL_COLOR_PG
            sta SYS_CTRL_1

            lda #$F0
            sta $c000

            ; Go back to I/O page 0
            stz SYS_CTRL_1

            ; Make sure we're in text mode
            lda #$01            ; enable TEXT
            sta $D000           ; Save that to VICKY master control register 0
            stz $D001

            ; Re-enable IRQ handling
            cli
\end{verbatim}

To actually process the interrupt, we need to read and then increment the character at the start of the screen, clear the pending flag for the SOF interrupt, and then return. However, the screen and the interrupt control registers are in different I/O banks, so we'll need to change the I/O bank a couple of times during interrupt processing. So, the first thing we will do is to save the value of the system control register at 0x0001, so we can restore it before we return from the interrupt.

\begin{verbatim}
SYS_CTRL_1 = $0001
SYS_CTRL_TEXT_PG = $02

my_handler: pha

            ; Save the system control register
            lda SYS_CTRL_1
            pha

            ; Switch to I/O page 0
            stz SYS_CTRL_1

            ; Check for SOF flag
            lda #INT00_VKY_SOF
            bit INT_PEND_0
            beq return              ; If it's zero, exit the handler

            ; Yes: clear the flag for SOF
            sta INT_PEND_0

            ; Move to the text screen page
            lda #SYS_CTRL_TEXT_PG
            sta SYS_CTRL_1

            ; Increment the character at position 0
            inc $c000

            ; Restore the system control register
return:     pla
            sta SYS_CTRL_1

            ; Return to the original code
            pla
            rti
\end{verbatim}
