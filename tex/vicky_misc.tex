\chapter{Miscellaneous Features of TinyVicky}

\section*{DIP Switches}

\jr\ has eight DIP switches on the board, which can be used to configure various options. The DIP switches are present on a single register (see table:~\ref{tab:dip_reg}). The DIP switches ground their signals when placed in their ``on'' positions. So a true or asserted value is 0, while the false or de-asserted value is 1.

\begin{table}[ht]
    \begin{center}
        \begin{tabular}{|c|c|c|c|c|c|c|c|c|c|} \hline
            Address & R/W & 7 & 6 & 5 & 4 & 3 & 2 & 1 & 0 \\\hline\hline
            \verb+0xD670+ & R & GAMMA & USER2 & USER1 & USER0 & \multicolumn{4}{|c|}{BOOT} \\ \hline
        \end{tabular}
    \end{center}
    \caption{DIP Switch Register}
    \label{tab:dip_reg}
\end{table}

There are five fields of switches:
\begin{description}
    \item[GAMMA] this is a dedicated switch to indicate if gamma correction should be turned on (0) or not (1)

    \item[USER0, USER1, USER2] these three switches are reserved for use by the operating system or programs. On is 0, off is 1.

    \item[BOOT] these four switches provide information to the operating system for boot options.
\end{description}

\section*{The Border}

The \jr's display can have a border, which overlays all the other display elements. The border can have any color which TinyVicky can display, and can have a width from 0 to 31 pixels. The border can also be turned off, leaving the full display for graphics or text.

When using graphics modes, the border simply hides the graphics elements underneath it. For text mode, things are a little different. The text display will be shifted so that the character at $(0, 0)$ is still the upper-left character. The layout of the text and color matrixes do not change, however. Cells that are under the right side or bottom of the border will still be in the matrixes but will not be displayed. Another way to put it is that, if the text resolution is 80 characters wide, it will remain 80 characters per line even if the border is on and only 76 characters are displayed.

\begin{table}[ht]
    \begin{center}
        \begin{tabular}{|c|c|c|c|c|c|c|c|c|c|c|} \hline
            Address & R/W & Name & 7 & 6 & 5 & 4 & 3 & 2 & 1 & 0 \\\hline\hline
            \verb+0xD004+ & R/W & BRDR\_CTRL & --- & \multicolumn{3}{|c|}{SCROLL\_X} & \multicolumn{3}{|c|}{---} & ENABLE \\ \hline
            \verb+0xD005+ & R/W & BRDR\_BLUE & \multicolumn{8}{|c|}{Blue component of border color} \\ \hline
            \verb+0xD006+ & R/W & BRDR\_GREEN & \multicolumn{8}{|c|}{Green component of border color} \\ \hline
            \verb+0xD007+ & R/W & BRDR\_RED & \multicolumn{8}{|c|}{Red component of border color} \\ \hline
            \verb+0xD008+ & R/W & BRDR\_WIDTH & \multicolumn{3}{|c|}{---} & \multicolumn{5}{|c|}{SIZE\_X} \\ \hline
            \verb+0xD009+ & R/W & BRDR\_HEIGHT & \multicolumn{3}{|c|}{---} & \multicolumn{5}{|c|}{SIZE\_Y} \\ \hline
        \end{tabular}
    \end{center}
    \caption{Border Registers}
    \label{tab:brdr_reg}
\end{table}

\begin{description}
    \item[ENABLE] when set (1), the border will be displayed

    \item[SCROLL\_X] the number of pixels the border should be shifted in the horizontal direction

    \item[BRDR\_BLUE] the amount of blue in the border (0 = none, 255 = maximum amount)

    \item[BRDR\_GREEN] the amount of green in the border (0 = none, 255 = maximum amount)

    \item[BRDR\_RED] the amount of red in the border (0 = none, 255 = maximum amount)

    \item[SIZE\_X] the width of the left and right sides of the border in pixels (from 0 to 31)

    \item[SIZE\_X] the height of top and bottom of the border in pixels (from 0 to 31)
\end{description}

\section*{Background Color}

In text mode, the background color is determined by the color matrix and the text color LUTs. For the graphics modes, however, a background color is specified separately. There are three registers to specify the background color's red, green, and blue components (see table:~\ref{tab:back_reg}). This is the color that will be displayed in graphics modes, if all the layers specify that a given pixel has the color 0 (which is always the transparent pixel color).

\begin{table}[ht]
    \begin{center}
        \begin{tabular}{|c|c|c|c|c|c|c|c|c|c|c|} \hline
            Address & R/W & Name & 7 & 6 & 5 & 4 & 3 & 2 & 1 & 0 \\\hline\hline
            \verb+0xD00D+ & R/W & BGND\_BLUE & \multicolumn{8}{|c|}{Blue component of background color} \\ \hline
            \verb+0xD00E+ & R/W & BGND\_GREEN & \multicolumn{8}{|c|}{Green component of background color} \\ \hline
            \verb+0xD00F+ & R/W & BGND\_RED & \multicolumn{8}{|c|}{Red component of background color} \\ \hline
        \end{tabular}
    \end{center}
    \caption{Background Color Registers}
    \label{tab:back_reg}
\end{table}

\section*{Line Interrupt and Beam Position}

TinyVicky can trigger a SOL interrupt (see table:~\ref{tab:interrupts}) when the display has reached a given raster line. This can be useful for split-screen style effects or other programming tricks that work off of partitioning the screen into separate areas. To use this feature, a program would enable the line interrupt and set a register to the number of the line on the screen when the interrupt should be triggered. In addition to setting a line interrupt, there are two 12-bit registers that allow the program to see what line and column is TinyVicky is currently drawing. The addresses for all these registers overlap. The line interrupt registers are write-only, and the current beam position registers are read only (see table:~\ref{tab:lint_reg})

\begin{table}[ht]
    \begin{center}
        \begin{tabular}{|c|c|c|c|c|c|c|c|c|c|c|} \hline
            Address & R/W & Name & 7 & 6 & 5 & 4 & 3 & 2 & 1 & 0 \\\hline\hline
            \verb+0xD018+ & W & LINT\_CTRL & \multicolumn{7}{|c|}{---} & ENABLE \\ \hline
            \verb+0xD019+ & W & \multirow{2}{*}{LINT\_L} & L7 & L6 & L5 & L4 & L3 & L2 & L1 & L0 \\ \cline{1-2}\cline{4-11}
            \verb+0xD01A+ & W &  & \multicolumn{4}{|c|}{---} & L11 & L10 & L9 & L8 \\ \hline
            \verb+0xD01B+ & W & --- & \multicolumn{8}{|c|}{Reserved} \\ \hline

            \verb+0xD018+ & R & \multirow{2}{*}{RAST\_COL} & X7 & X6 & X5 & X4 & X3 & X2 & X1 & X0 \\  \cline{1-2}\cline{4-11}
            \verb+0xD019+ & R & & \multicolumn{4}{|c|}{---} & X11 & X10 & X9 & X8 \\ \hline
            \verb+0xD01A+ & R & \multirow{2}{*}{RAST\_ROW} & Y7 & Y6 & Y5 & Y4 & Y3 & Y2 & Y1 & Y0 \\  \cline{1-2}\cline{4-11}
            \verb+0xD01B+ & R & & \multicolumn{4}{|c|}{---} & Y11 & Y10 & Y9 & Y8 \\ \hline
        \end{tabular}
    \end{center}
    \caption{Line Interrupt and Beam Position Registers}
    \label{tab:lint_reg}
\end{table}

\begin{description}
    \item[ENABLE] if set (1), TinyVicky will trigger line interrupts (write only)

    \item[LINT\_L] the line number (12 bits) on which to trigger the next line interrupt (write only). The top of the display is line 0, and the bottom of the screen is 400 for $320 \times 200$ mode, and 480 for $320 \times 240$ mode.

    \item[RAST\_COL] the number (12 bits) of the current pixel being drawn (read only)

    \item[RAST\_ROW] the number (12 bits) of the current line being drawn (read only)
\end{description}

\example{Changing Border with the Line}
In this example, we will play with a split-screen style effect changing the color of the border so that the top and bottom borders are blue while the left and right borders are red. To do this, we will use the line interrupt twice for each frame: once when we are on the line just below the last line of the top border, and once when we are on the first line of the bottom border.

To make this work, the example has a single \verb+state+ variable, which will track which color border is being rendered. It will enable the line interrupt, and then set the line number to wait for. When that interrupt comes in, it will check the state variable, setting the border color and new line number based on \verb+state+. It will also flip \verb+state+ to the other value (0 or 1).

\begin{verbatim}
INT_PEND_0 = $D660              ; Pending register for interrupts 0 - 7
INT_PEND_1 = $D661              ; Pending register for interrupts 8 - 15
INT_MASK_0 = $D666              ; Mask register for interrupts 0 - 7
INT_MASK_1 = $D667              ; Mask register for interrupts 8 - 15
INT01_VKY_SOL = $02

MMU_IO_CTRL = $0001             ; MMU I/O Control Register

VKY_MSTR_CTRL_0 = $D000         ; Vicky Master Control Register 0
VKY_MSTR_CTRL_1 = $D001         ; Vicky Master Control Register 1
VKY_BRDR_CTRL = $D004           ; Vicky Border Control Register
VKY_BRDR_COL_B = $D005          ; Vicky Border Color -- Blue
VKY_BRDR_COL_G = $D006          ; Vicky Border Color -- Green
VKY_BRDR_COL_R = $D007          ; Vicky Border Color -- Red
VKY_BRDR_VERT = $D008           ; Vicky Border vertical thickness in pixels
VKY_BRDR_HORI = $D009           ; Vicky Border Horizontal Thickness in pixels

VIRQ = $FFFE

LINE0 = 16                      ; Start at line 16 (first line on the text display)
LINE1 = 480 - 16                ; End on line 464 (last line of text display)

;
; Variables
;
* = $0080

state       .byte ?             ; Variable to track which color we should use

;
; Code
;
* = $e000

start:      ; Disable IRQ handling
            sei

            ; Go back to I/O page 0
            stz MMU_IO_CTRL

            ; Load my IRQ handler into the IRQ vector
            ; NOTE: this code just takes over IRQs completely. It could save
            ;       the pointer to the old handler and chain to it when it had
            ;       handled its interrupt. But what is proper really depends on
            ;       what the program is trying to do.
            lda #<my_handler
            sta VIRQ
            lda #>my_handler
            sta VIRQ+1

            ; Mask off all but the SOL interrupt
            lda #$ff
            sta INT_MASK_1
            and #~INT01_VKY_SOL
            sta INT_MASK_0

            ; Clear all pending interrupts
            lda #$ff
            sta INT_PEND_0
            sta INT_PEND_1

            ; Make sure we're in text mode
            lda #$01                ; enable TEXT
            sta VKY_MSTR_CTRL_0     ; Save that to VICKY master control register 0
            stz VKY_MSTR_CTRL_1

            ; Set the border
            lda #$01                ; Enable the border
            sta VKY_BRDR_CTRL

            lda #16                 ; Make it 16 pixels wide
            sta VKY_BRDR_VERT
            sta VKY_BRDR_HORI

            lda #$80                ; Make it cyan to start with
            sta VKY_BRDR_COL_B
            sta VKY_BRDR_COL_G
            stz VKY_BRDR_COL_R

            lda #$01                ; Turn on the line interrupt
            sta VKY_LINE_CTRL

            lda #<LINE0             ; set the line to interrupt on
            sta VKY_LINE_NBR_L
            lda #>LINE0
            sta VKY_LINE_NBR_H

            stz state               ; Start in state 0

            ; Re-enable IRQ handling
            cli

loop:       ; Just loop forever... a real program will do stuff here
            nop
            bra loop

;
; A simple interrupt handler
;
my_handler: .proc
            pha

            ; Save the system control register
            lda MMU_IO_CTRL
            pha

            ; Switch to I/O page 0
            stz MMU_IO_CTRL

            ; Check for SOL flag
            lda #INT01_VKY_SOL
            bit INT_PEND_0
            beq return              ; If it's zero, exit the handler

            ; Yes: clear the flag for SOL
            sta INT_PEND_0

            lda state               ; Check the state
            beq is_zero

            stz state               ; If state 1: Set the state to 0

            lda #<LINE0             ; Set the line to interrupt on
            sta VKY_LINE_NBR_L
            lda #>LINE0
            sta VKY_LINE_NBR_H

            lda #$80                ; Make the border blue
            sta VKY_BRDR_COL_B
            stz VKY_BRDR_COL_G
            stz VKY_BRDR_COL_R
            bra return

is_zero:    lda #$01                ; Set the state to 1
            sta state

            lda #<LINE1             ; set the line to interrupt on
            sta VKY_LINE_NBR_L
            lda #>LINE1
            sta VKY_LINE_NBR_H

            lda #$80                ; Make the border red
            sta VKY_BRDR_COL_R
            stz VKY_BRDR_COL_G
            stz VKY_BRDR_COL_B

            ; Restore the system control register
return:     pla
            sta MMU_IO_CTRL

            ; Return to the original code
            pla
            rti
            .pend
\end{verbatim}

\section*{Gamma Correction}

TinyVicky has the ability to apply gamma correction to the video signal. This allows users to adjust their images to match their monitors. Activating gamma correction is done by setting the GAMMA flag in the Vicky master control register (see table:~\ref{tab:vky_master_ctrl_reg}). When enabled, colors will be adjusted through the gamma look up tables. There are three tables: blue is at 0xC000, green is at 0xC400, and red is at 0xC800.

The way that the gamma look up tables work is very straight forward. When drawing a pixel, the separate color components are used as indexes into their respective gamma LUTs, and the value in the LUT is used as the new component value. For instance, if a pixel's color is $(r, g, b)$, then the new color is:
\begin{verbatim}
    r_corrected = gamma_red[r]
    g_corrected = gamma_green[g]
    b_corrected = gamma_blue[b]
\end{verbatim}

On power up, TinyVicky sets up a default gamma correction of 1.8, although software (either the user's program or the operating system) has to turn on gamma correction to use it.
